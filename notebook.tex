
% Default to the notebook output style

    


% Inherit from the specified cell style.




    
\documentclass[11pt]{article}

    
    
    \usepackage[T1]{fontenc}
    % Nicer default font (+ math font) than Computer Modern for most use cases
    \usepackage{mathpazo}

    % Basic figure setup, for now with no caption control since it's done
    % automatically by Pandoc (which extracts ![](path) syntax from Markdown).
    \usepackage{graphicx}
    % We will generate all images so they have a width \maxwidth. This means
    % that they will get their normal width if they fit onto the page, but
    % are scaled down if they would overflow the margins.
    \makeatletter
    \def\maxwidth{\ifdim\Gin@nat@width>\linewidth\linewidth
    \else\Gin@nat@width\fi}
    \makeatother
    \let\Oldincludegraphics\includegraphics
    % Set max figure width to be 80% of text width, for now hardcoded.
    \renewcommand{\includegraphics}[1]{\Oldincludegraphics[width=.8\maxwidth]{#1}}
    % Ensure that by default, figures have no caption (until we provide a
    % proper Figure object with a Caption API and a way to capture that
    % in the conversion process - todo).
    \usepackage{caption}
    \DeclareCaptionLabelFormat{nolabel}{}
    \captionsetup{labelformat=nolabel}

    \usepackage{adjustbox} % Used to constrain images to a maximum size 
    \usepackage{xcolor} % Allow colors to be defined
    \usepackage{enumerate} % Needed for markdown enumerations to work
    \usepackage{geometry} % Used to adjust the document margins
    \usepackage{amsmath} % Equations
    \usepackage{amssymb} % Equations
    \usepackage{textcomp} % defines textquotesingle
    % Hack from http://tex.stackexchange.com/a/47451/13684:
    \AtBeginDocument{%
        \def\PYZsq{\textquotesingle}% Upright quotes in Pygmentized code
    }
    \usepackage{upquote} % Upright quotes for verbatim code
    \usepackage{eurosym} % defines \euro
    \usepackage[mathletters]{ucs} % Extended unicode (utf-8) support
    \usepackage[utf8x]{inputenc} % Allow utf-8 characters in the tex document
    \usepackage{fancyvrb} % verbatim replacement that allows latex
    \usepackage{grffile} % extends the file name processing of package graphics 
                         % to support a larger range 
    % The hyperref package gives us a pdf with properly built
    % internal navigation ('pdf bookmarks' for the table of contents,
    % internal cross-reference links, web links for URLs, etc.)
    \usepackage{hyperref}
    \usepackage{longtable} % longtable support required by pandoc >1.10
    \usepackage{booktabs}  % table support for pandoc > 1.12.2
    \usepackage[inline]{enumitem} % IRkernel/repr support (it uses the enumerate* environment)
    \usepackage[normalem]{ulem} % ulem is needed to support strikethroughs (\sout)
                                % normalem makes italics be italics, not underlines
    

    
    
    % Colors for the hyperref package
    \definecolor{urlcolor}{rgb}{0,.145,.698}
    \definecolor{linkcolor}{rgb}{.71,0.21,0.01}
    \definecolor{citecolor}{rgb}{.12,.54,.11}

    % ANSI colors
    \definecolor{ansi-black}{HTML}{3E424D}
    \definecolor{ansi-black-intense}{HTML}{282C36}
    \definecolor{ansi-red}{HTML}{E75C58}
    \definecolor{ansi-red-intense}{HTML}{B22B31}
    \definecolor{ansi-green}{HTML}{00A250}
    \definecolor{ansi-green-intense}{HTML}{007427}
    \definecolor{ansi-yellow}{HTML}{DDB62B}
    \definecolor{ansi-yellow-intense}{HTML}{B27D12}
    \definecolor{ansi-blue}{HTML}{208FFB}
    \definecolor{ansi-blue-intense}{HTML}{0065CA}
    \definecolor{ansi-magenta}{HTML}{D160C4}
    \definecolor{ansi-magenta-intense}{HTML}{A03196}
    \definecolor{ansi-cyan}{HTML}{60C6C8}
    \definecolor{ansi-cyan-intense}{HTML}{258F8F}
    \definecolor{ansi-white}{HTML}{C5C1B4}
    \definecolor{ansi-white-intense}{HTML}{A1A6B2}

    % commands and environments needed by pandoc snippets
    % extracted from the output of `pandoc -s`
    \providecommand{\tightlist}{%
      \setlength{\itemsep}{0pt}\setlength{\parskip}{0pt}}
    \DefineVerbatimEnvironment{Highlighting}{Verbatim}{commandchars=\\\{\}}
    % Add ',fontsize=\small' for more characters per line
    \newenvironment{Shaded}{}{}
    \newcommand{\KeywordTok}[1]{\textcolor[rgb]{0.00,0.44,0.13}{\textbf{{#1}}}}
    \newcommand{\DataTypeTok}[1]{\textcolor[rgb]{0.56,0.13,0.00}{{#1}}}
    \newcommand{\DecValTok}[1]{\textcolor[rgb]{0.25,0.63,0.44}{{#1}}}
    \newcommand{\BaseNTok}[1]{\textcolor[rgb]{0.25,0.63,0.44}{{#1}}}
    \newcommand{\FloatTok}[1]{\textcolor[rgb]{0.25,0.63,0.44}{{#1}}}
    \newcommand{\CharTok}[1]{\textcolor[rgb]{0.25,0.44,0.63}{{#1}}}
    \newcommand{\StringTok}[1]{\textcolor[rgb]{0.25,0.44,0.63}{{#1}}}
    \newcommand{\CommentTok}[1]{\textcolor[rgb]{0.38,0.63,0.69}{\textit{{#1}}}}
    \newcommand{\OtherTok}[1]{\textcolor[rgb]{0.00,0.44,0.13}{{#1}}}
    \newcommand{\AlertTok}[1]{\textcolor[rgb]{1.00,0.00,0.00}{\textbf{{#1}}}}
    \newcommand{\FunctionTok}[1]{\textcolor[rgb]{0.02,0.16,0.49}{{#1}}}
    \newcommand{\RegionMarkerTok}[1]{{#1}}
    \newcommand{\ErrorTok}[1]{\textcolor[rgb]{1.00,0.00,0.00}{\textbf{{#1}}}}
    \newcommand{\NormalTok}[1]{{#1}}
    
    % Additional commands for more recent versions of Pandoc
    \newcommand{\ConstantTok}[1]{\textcolor[rgb]{0.53,0.00,0.00}{{#1}}}
    \newcommand{\SpecialCharTok}[1]{\textcolor[rgb]{0.25,0.44,0.63}{{#1}}}
    \newcommand{\VerbatimStringTok}[1]{\textcolor[rgb]{0.25,0.44,0.63}{{#1}}}
    \newcommand{\SpecialStringTok}[1]{\textcolor[rgb]{0.73,0.40,0.53}{{#1}}}
    \newcommand{\ImportTok}[1]{{#1}}
    \newcommand{\DocumentationTok}[1]{\textcolor[rgb]{0.73,0.13,0.13}{\textit{{#1}}}}
    \newcommand{\AnnotationTok}[1]{\textcolor[rgb]{0.38,0.63,0.69}{\textbf{\textit{{#1}}}}}
    \newcommand{\CommentVarTok}[1]{\textcolor[rgb]{0.38,0.63,0.69}{\textbf{\textit{{#1}}}}}
    \newcommand{\VariableTok}[1]{\textcolor[rgb]{0.10,0.09,0.49}{{#1}}}
    \newcommand{\ControlFlowTok}[1]{\textcolor[rgb]{0.00,0.44,0.13}{\textbf{{#1}}}}
    \newcommand{\OperatorTok}[1]{\textcolor[rgb]{0.40,0.40,0.40}{{#1}}}
    \newcommand{\BuiltInTok}[1]{{#1}}
    \newcommand{\ExtensionTok}[1]{{#1}}
    \newcommand{\PreprocessorTok}[1]{\textcolor[rgb]{0.74,0.48,0.00}{{#1}}}
    \newcommand{\AttributeTok}[1]{\textcolor[rgb]{0.49,0.56,0.16}{{#1}}}
    \newcommand{\InformationTok}[1]{\textcolor[rgb]{0.38,0.63,0.69}{\textbf{\textit{{#1}}}}}
    \newcommand{\WarningTok}[1]{\textcolor[rgb]{0.38,0.63,0.69}{\textbf{\textit{{#1}}}}}
    
    
    % Define a nice break command that doesn't care if a line doesn't already
    % exist.
    \def\br{\hspace*{\fill} \\* }
    % Math Jax compatability definitions
    \def\gt{>}
    \def\lt{<}
    % Document parameters
    \title{HAR\_EDA}
    
    
    

    % Pygments definitions
    
\makeatletter
\def\PY@reset{\let\PY@it=\relax \let\PY@bf=\relax%
    \let\PY@ul=\relax \let\PY@tc=\relax%
    \let\PY@bc=\relax \let\PY@ff=\relax}
\def\PY@tok#1{\csname PY@tok@#1\endcsname}
\def\PY@toks#1+{\ifx\relax#1\empty\else%
    \PY@tok{#1}\expandafter\PY@toks\fi}
\def\PY@do#1{\PY@bc{\PY@tc{\PY@ul{%
    \PY@it{\PY@bf{\PY@ff{#1}}}}}}}
\def\PY#1#2{\PY@reset\PY@toks#1+\relax+\PY@do{#2}}

\expandafter\def\csname PY@tok@w\endcsname{\def\PY@tc##1{\textcolor[rgb]{0.73,0.73,0.73}{##1}}}
\expandafter\def\csname PY@tok@c\endcsname{\let\PY@it=\textit\def\PY@tc##1{\textcolor[rgb]{0.25,0.50,0.50}{##1}}}
\expandafter\def\csname PY@tok@cp\endcsname{\def\PY@tc##1{\textcolor[rgb]{0.74,0.48,0.00}{##1}}}
\expandafter\def\csname PY@tok@k\endcsname{\let\PY@bf=\textbf\def\PY@tc##1{\textcolor[rgb]{0.00,0.50,0.00}{##1}}}
\expandafter\def\csname PY@tok@kp\endcsname{\def\PY@tc##1{\textcolor[rgb]{0.00,0.50,0.00}{##1}}}
\expandafter\def\csname PY@tok@kt\endcsname{\def\PY@tc##1{\textcolor[rgb]{0.69,0.00,0.25}{##1}}}
\expandafter\def\csname PY@tok@o\endcsname{\def\PY@tc##1{\textcolor[rgb]{0.40,0.40,0.40}{##1}}}
\expandafter\def\csname PY@tok@ow\endcsname{\let\PY@bf=\textbf\def\PY@tc##1{\textcolor[rgb]{0.67,0.13,1.00}{##1}}}
\expandafter\def\csname PY@tok@nb\endcsname{\def\PY@tc##1{\textcolor[rgb]{0.00,0.50,0.00}{##1}}}
\expandafter\def\csname PY@tok@nf\endcsname{\def\PY@tc##1{\textcolor[rgb]{0.00,0.00,1.00}{##1}}}
\expandafter\def\csname PY@tok@nc\endcsname{\let\PY@bf=\textbf\def\PY@tc##1{\textcolor[rgb]{0.00,0.00,1.00}{##1}}}
\expandafter\def\csname PY@tok@nn\endcsname{\let\PY@bf=\textbf\def\PY@tc##1{\textcolor[rgb]{0.00,0.00,1.00}{##1}}}
\expandafter\def\csname PY@tok@ne\endcsname{\let\PY@bf=\textbf\def\PY@tc##1{\textcolor[rgb]{0.82,0.25,0.23}{##1}}}
\expandafter\def\csname PY@tok@nv\endcsname{\def\PY@tc##1{\textcolor[rgb]{0.10,0.09,0.49}{##1}}}
\expandafter\def\csname PY@tok@no\endcsname{\def\PY@tc##1{\textcolor[rgb]{0.53,0.00,0.00}{##1}}}
\expandafter\def\csname PY@tok@nl\endcsname{\def\PY@tc##1{\textcolor[rgb]{0.63,0.63,0.00}{##1}}}
\expandafter\def\csname PY@tok@ni\endcsname{\let\PY@bf=\textbf\def\PY@tc##1{\textcolor[rgb]{0.60,0.60,0.60}{##1}}}
\expandafter\def\csname PY@tok@na\endcsname{\def\PY@tc##1{\textcolor[rgb]{0.49,0.56,0.16}{##1}}}
\expandafter\def\csname PY@tok@nt\endcsname{\let\PY@bf=\textbf\def\PY@tc##1{\textcolor[rgb]{0.00,0.50,0.00}{##1}}}
\expandafter\def\csname PY@tok@nd\endcsname{\def\PY@tc##1{\textcolor[rgb]{0.67,0.13,1.00}{##1}}}
\expandafter\def\csname PY@tok@s\endcsname{\def\PY@tc##1{\textcolor[rgb]{0.73,0.13,0.13}{##1}}}
\expandafter\def\csname PY@tok@sd\endcsname{\let\PY@it=\textit\def\PY@tc##1{\textcolor[rgb]{0.73,0.13,0.13}{##1}}}
\expandafter\def\csname PY@tok@si\endcsname{\let\PY@bf=\textbf\def\PY@tc##1{\textcolor[rgb]{0.73,0.40,0.53}{##1}}}
\expandafter\def\csname PY@tok@se\endcsname{\let\PY@bf=\textbf\def\PY@tc##1{\textcolor[rgb]{0.73,0.40,0.13}{##1}}}
\expandafter\def\csname PY@tok@sr\endcsname{\def\PY@tc##1{\textcolor[rgb]{0.73,0.40,0.53}{##1}}}
\expandafter\def\csname PY@tok@ss\endcsname{\def\PY@tc##1{\textcolor[rgb]{0.10,0.09,0.49}{##1}}}
\expandafter\def\csname PY@tok@sx\endcsname{\def\PY@tc##1{\textcolor[rgb]{0.00,0.50,0.00}{##1}}}
\expandafter\def\csname PY@tok@m\endcsname{\def\PY@tc##1{\textcolor[rgb]{0.40,0.40,0.40}{##1}}}
\expandafter\def\csname PY@tok@gh\endcsname{\let\PY@bf=\textbf\def\PY@tc##1{\textcolor[rgb]{0.00,0.00,0.50}{##1}}}
\expandafter\def\csname PY@tok@gu\endcsname{\let\PY@bf=\textbf\def\PY@tc##1{\textcolor[rgb]{0.50,0.00,0.50}{##1}}}
\expandafter\def\csname PY@tok@gd\endcsname{\def\PY@tc##1{\textcolor[rgb]{0.63,0.00,0.00}{##1}}}
\expandafter\def\csname PY@tok@gi\endcsname{\def\PY@tc##1{\textcolor[rgb]{0.00,0.63,0.00}{##1}}}
\expandafter\def\csname PY@tok@gr\endcsname{\def\PY@tc##1{\textcolor[rgb]{1.00,0.00,0.00}{##1}}}
\expandafter\def\csname PY@tok@ge\endcsname{\let\PY@it=\textit}
\expandafter\def\csname PY@tok@gs\endcsname{\let\PY@bf=\textbf}
\expandafter\def\csname PY@tok@gp\endcsname{\let\PY@bf=\textbf\def\PY@tc##1{\textcolor[rgb]{0.00,0.00,0.50}{##1}}}
\expandafter\def\csname PY@tok@go\endcsname{\def\PY@tc##1{\textcolor[rgb]{0.53,0.53,0.53}{##1}}}
\expandafter\def\csname PY@tok@gt\endcsname{\def\PY@tc##1{\textcolor[rgb]{0.00,0.27,0.87}{##1}}}
\expandafter\def\csname PY@tok@err\endcsname{\def\PY@bc##1{\setlength{\fboxsep}{0pt}\fcolorbox[rgb]{1.00,0.00,0.00}{1,1,1}{\strut ##1}}}
\expandafter\def\csname PY@tok@kc\endcsname{\let\PY@bf=\textbf\def\PY@tc##1{\textcolor[rgb]{0.00,0.50,0.00}{##1}}}
\expandafter\def\csname PY@tok@kd\endcsname{\let\PY@bf=\textbf\def\PY@tc##1{\textcolor[rgb]{0.00,0.50,0.00}{##1}}}
\expandafter\def\csname PY@tok@kn\endcsname{\let\PY@bf=\textbf\def\PY@tc##1{\textcolor[rgb]{0.00,0.50,0.00}{##1}}}
\expandafter\def\csname PY@tok@kr\endcsname{\let\PY@bf=\textbf\def\PY@tc##1{\textcolor[rgb]{0.00,0.50,0.00}{##1}}}
\expandafter\def\csname PY@tok@bp\endcsname{\def\PY@tc##1{\textcolor[rgb]{0.00,0.50,0.00}{##1}}}
\expandafter\def\csname PY@tok@fm\endcsname{\def\PY@tc##1{\textcolor[rgb]{0.00,0.00,1.00}{##1}}}
\expandafter\def\csname PY@tok@vc\endcsname{\def\PY@tc##1{\textcolor[rgb]{0.10,0.09,0.49}{##1}}}
\expandafter\def\csname PY@tok@vg\endcsname{\def\PY@tc##1{\textcolor[rgb]{0.10,0.09,0.49}{##1}}}
\expandafter\def\csname PY@tok@vi\endcsname{\def\PY@tc##1{\textcolor[rgb]{0.10,0.09,0.49}{##1}}}
\expandafter\def\csname PY@tok@vm\endcsname{\def\PY@tc##1{\textcolor[rgb]{0.10,0.09,0.49}{##1}}}
\expandafter\def\csname PY@tok@sa\endcsname{\def\PY@tc##1{\textcolor[rgb]{0.73,0.13,0.13}{##1}}}
\expandafter\def\csname PY@tok@sb\endcsname{\def\PY@tc##1{\textcolor[rgb]{0.73,0.13,0.13}{##1}}}
\expandafter\def\csname PY@tok@sc\endcsname{\def\PY@tc##1{\textcolor[rgb]{0.73,0.13,0.13}{##1}}}
\expandafter\def\csname PY@tok@dl\endcsname{\def\PY@tc##1{\textcolor[rgb]{0.73,0.13,0.13}{##1}}}
\expandafter\def\csname PY@tok@s2\endcsname{\def\PY@tc##1{\textcolor[rgb]{0.73,0.13,0.13}{##1}}}
\expandafter\def\csname PY@tok@sh\endcsname{\def\PY@tc##1{\textcolor[rgb]{0.73,0.13,0.13}{##1}}}
\expandafter\def\csname PY@tok@s1\endcsname{\def\PY@tc##1{\textcolor[rgb]{0.73,0.13,0.13}{##1}}}
\expandafter\def\csname PY@tok@mb\endcsname{\def\PY@tc##1{\textcolor[rgb]{0.40,0.40,0.40}{##1}}}
\expandafter\def\csname PY@tok@mf\endcsname{\def\PY@tc##1{\textcolor[rgb]{0.40,0.40,0.40}{##1}}}
\expandafter\def\csname PY@tok@mh\endcsname{\def\PY@tc##1{\textcolor[rgb]{0.40,0.40,0.40}{##1}}}
\expandafter\def\csname PY@tok@mi\endcsname{\def\PY@tc##1{\textcolor[rgb]{0.40,0.40,0.40}{##1}}}
\expandafter\def\csname PY@tok@il\endcsname{\def\PY@tc##1{\textcolor[rgb]{0.40,0.40,0.40}{##1}}}
\expandafter\def\csname PY@tok@mo\endcsname{\def\PY@tc##1{\textcolor[rgb]{0.40,0.40,0.40}{##1}}}
\expandafter\def\csname PY@tok@ch\endcsname{\let\PY@it=\textit\def\PY@tc##1{\textcolor[rgb]{0.25,0.50,0.50}{##1}}}
\expandafter\def\csname PY@tok@cm\endcsname{\let\PY@it=\textit\def\PY@tc##1{\textcolor[rgb]{0.25,0.50,0.50}{##1}}}
\expandafter\def\csname PY@tok@cpf\endcsname{\let\PY@it=\textit\def\PY@tc##1{\textcolor[rgb]{0.25,0.50,0.50}{##1}}}
\expandafter\def\csname PY@tok@c1\endcsname{\let\PY@it=\textit\def\PY@tc##1{\textcolor[rgb]{0.25,0.50,0.50}{##1}}}
\expandafter\def\csname PY@tok@cs\endcsname{\let\PY@it=\textit\def\PY@tc##1{\textcolor[rgb]{0.25,0.50,0.50}{##1}}}

\def\PYZbs{\char`\\}
\def\PYZus{\char`\_}
\def\PYZob{\char`\{}
\def\PYZcb{\char`\}}
\def\PYZca{\char`\^}
\def\PYZam{\char`\&}
\def\PYZlt{\char`\<}
\def\PYZgt{\char`\>}
\def\PYZsh{\char`\#}
\def\PYZpc{\char`\%}
\def\PYZdl{\char`\$}
\def\PYZhy{\char`\-}
\def\PYZsq{\char`\'}
\def\PYZdq{\char`\"}
\def\PYZti{\char`\~}
% for compatibility with earlier versions
\def\PYZat{@}
\def\PYZlb{[}
\def\PYZrb{]}
\makeatother


    % Exact colors from NB
    \definecolor{incolor}{rgb}{0.0, 0.0, 0.5}
    \definecolor{outcolor}{rgb}{0.545, 0.0, 0.0}



    
    % Prevent overflowing lines due to hard-to-break entities
    \sloppy 
    % Setup hyperref package
    \hypersetup{
      breaklinks=true,  % so long urls are correctly broken across lines
      colorlinks=true,
      urlcolor=urlcolor,
      linkcolor=linkcolor,
      citecolor=citecolor,
      }
    % Slightly bigger margins than the latex defaults
    
    \geometry{verbose,tmargin=1in,bmargin=1in,lmargin=1in,rmargin=1in}
    
    

    \begin{document}
    
    
    \maketitle
    
    

    
    \hypertarget{humanactivityrecognition}{%
\section{HumanActivityRecognition}\label{humanactivityrecognition}}

This project is to build a model that predicts the human activities such
as Walking, Walking\_Upstairs, Walking\_Downstairs, Sitting, Standing or
Laying.

This dataset is collected from 30 persons(referred as subjects in this
dataset), performing different activities with a smartphone to their
waists. The data is recorded with the help of sensors (accelerometer and
Gyroscope) in that smartphone. This experiment was video recorded to
label the data manually.

\hypertarget{how-data-was-recorded}{%
\subsection{How data was recorded}\label{how-data-was-recorded}}

By using the sensors(Gyroscope and accelerometer) in a smartphone, they
have captured `3-axial linear acceleration'(\emph{tAcc-XYZ}) from
accelerometer and `3-axial angular velocity' (\emph{tGyro-XYZ}) from
Gyroscope with several variations.

\begin{quote}
prefix `t' in those metrics denotes time.
\end{quote}

\begin{quote}
suffix `XYZ' represents 3-axial signals in X , Y, and Z directions.
\end{quote}

\hypertarget{feature-names}{%
\subsubsection{Feature names}\label{feature-names}}

\begin{enumerate}
\def\labelenumi{\arabic{enumi}.}
\item
  These sensor signals are preprocessed by applying noise filters and
  then sampled in fixed-width windows(sliding windows) of 2.56 seconds
  each with 50\% overlap. ie., each window has 128 readings.
\item
  From Each window, a feature vector was obtianed by calculating
  variables from the time and frequency domain. \textgreater{} In our
  dataset, each datapoint represents a window with different readings
\item
  The accelertion signal was saperated into Body and Gravity
  acceleration signals(\textbf{\emph{tBodyAcc-XYZ}} and
  \textbf{\emph{tGravityAcc-XYZ}}) using some low pass filter with
  corner frequecy of 0.3Hz.
\item
  After that, the body linear acceleration and angular velocity were
  derived in time to obtian \emph{jerk signals}
  (\textbf{\emph{tBodyAccJerk-XYZ}} and
  \textbf{\emph{tBodyGyroJerk-XYZ}}).
\item
  The magnitude of these 3-dimensional signals were calculated using the
  Euclidian norm. This magnitudes are represented as features with names
  like \emph{tBodyAccMag}, \emph{tGravityAccMag},
  \emph{tBodyAccJerkMag}, \emph{tBodyGyroMag} and
  \emph{tBodyGyroJerkMag}.
\item
  Finally, We've got frequency domain signals from some of the available
  signals by applying a FFT (Fast Fourier Transform). These signals
  obtained were labeled with \textbf{\emph{prefix `f'}} just like
  original signals with \textbf{\emph{prefix `t'}}. These signals are
  labeled as \textbf{\emph{fBodyAcc-XYZ}}, \textbf{\emph{fBodyGyroMag}}
  etc.,.
\item
  These are the signals that we got so far.

  \begin{itemize}
  \tightlist
  \item
    tBodyAcc-XYZ
  \item
    tGravityAcc-XYZ
  \item
    tBodyAccJerk-XYZ
  \item
    tBodyGyro-XYZ
  \item
    tBodyGyroJerk-XYZ
  \item
    tBodyAccMag
  \item
    tGravityAccMag
  \item
    tBodyAccJerkMag
  \item
    tBodyGyroMag
  \item
    tBodyGyroJerkMag
  \item
    fBodyAcc-XYZ
  \item
    fBodyAccJerk-XYZ
  \item
    fBodyGyro-XYZ
  \item
    fBodyAccMag
  \item
    fBodyAccJerkMag
  \item
    fBodyGyroMag
  \item
    fBodyGyroJerkMag
  \end{itemize}
\item
  We can esitmate some set of variables from the above signals. ie., We
  will estimate the following properties on each and every signal that
  we recoreded so far.

  \begin{itemize}
  \tightlist
  \item
    \textbf{\emph{mean()}}: Mean value
  \item
    \textbf{\emph{std()}}: Standard deviation
  \item
    \textbf{\emph{mad()}}: Median absolute deviation
  \item
    \textbf{\emph{max()}}: Largest value in array
  \item
    \textbf{\emph{min()}}: Smallest value in array
  \item
    \textbf{\emph{sma()}}: Signal magnitude area
  \item
    \textbf{\emph{energy()}}: Energy measure. Sum of the squares divided
    by the number of values.
  \item
    \textbf{\emph{iqr()}}: Interquartile range
  \item
    \textbf{\emph{entropy()}}: Signal entropy
  \item
    \textbf{\emph{arCoeff()}}: Autorregresion coefficients with Burg
    order equal to 4
  \item
    \textbf{\emph{correlation()}}: correlation coefficient between two
    signals
  \item
    \textbf{\emph{maxInds()}}: index of the frequency component with
    largest magnitude
  \item
    \textbf{\emph{meanFreq()}}: Weighted average of the frequency
    components to obtain a mean frequency
  \item
    \textbf{\emph{skewness()}}: skewness of the frequency domain signal
  \item
    \textbf{\emph{kurtosis()}}: kurtosis of the frequency domain signal
  \item
    \textbf{\emph{bandsEnergy()}}: Energy of a frequency interval within
    the 64 bins of the FFT of each window.
  \item
    \textbf{\emph{angle()}}: Angle between to vectors.
  \end{itemize}
\item
  We can obtain some other vectors by taking the average of signals in a
  single window sample. These are used on the angle() variable' `

  \begin{itemize}
  \tightlist
  \item
    gravityMean
  \item
    tBodyAccMean
  \item
    tBodyAccJerkMean
  \item
    tBodyGyroMean
  \item
    tBodyGyroJerkMean
  \end{itemize}
\end{enumerate}

\hypertarget{y_labelsencoded}{%
\subsubsection{Y\_Labels(Encoded)}\label{y_labelsencoded}}

\begin{itemize}
\item
  In the dataset, Y\_labels are represented as numbers from 1 to 6 as
  their identifiers.

  \begin{itemize}
  \tightlist
  \item
    WALKING as \textbf{1}
  \item
    WALKING\_UPSTAIRS as \textbf{2}
  \item
    WALKING\_DOWNSTAIRS as \textbf{3}
  \item
    SITTING as \textbf{4}
  \item
    STANDING as \textbf{5}
  \item
    LAYING as \textbf{6}
  \end{itemize}
\end{itemize}

\hypertarget{train-and-test-data-were-saperated}} of the volunteers were taken as
  \textbf{\emph{trianing data}} and remaining \textbf{\emph{30\%}}
  subjects recordings were taken for \textbf{\emph{test data}}
\end{itemize}

\hypertarget{data}{%
\subsection{Data}\label{data}}

\begin{itemize}
\tightlist
\item
  All the data is present in `UCI\_HAR\_dataset/' folder in present
  working directory.

  \begin{itemize}
  \tightlist
  \item
    Feature names are present in `UCI\_HAR\_dataset/features.txt'
  \item
    \textbf{\emph{Train Data}}

    \begin{itemize}
    \tightlist
    \item
      `UCI\_HAR\_dataset/train/X\_train.txt'
    \item
      `UCI\_HAR\_dataset/train/subject\_train.txt'
    \item
      `UCI\_HAR\_dataset/train/y\_train.txt'
    \end{itemize}
  \item
    \textbf{\emph{Test Data}}

    \begin{itemize}
    \tightlist
    \item
      `UCI\_HAR\_dataset/test/X\_test.txt'
    \item
      `UCI\_HAR\_dataset/test/subject\_test.txt'
    \item
      `UCI\_HAR\_dataset/test/y\_test.txt'
    \end{itemize}
  \end{itemize}
\end{itemize}

\hypertarget{data-size}{%
\subsection{Data Size :}\label{data-size}}

\begin{quote}
27 MB
\end{quote}

    

    

    

    \hypertarget{quick-overview-of-the-dataset}{%
\section{Quick overview of the dataset
:}\label{quick-overview-of-the-dataset}}

    \begin{itemize}
\item
  Accelerometer and Gyroscope readings are taken from 30
  volunteers(referred as subjects) while performing the following 6
  Activities.

  \begin{enumerate}
  \def\labelenumi{\arabic{enumi}.}
  \tightlist
  \item
    Walking\\
  \item
    WalkingUpstairs
  \item
    WalkingDownstairs
  \item
    Standing
  \item
    Sitting
  \item
    Lying.
  \end{enumerate}
\item
  Readings are divided into a window of 2.56 seconds with 50\%
  overlapping.
\item
  Accelerometer readings are divided into gravity acceleration and body
  acceleration readings, which has x,y and z components each.
\item
  Gyroscope readings are the measure of angular velocities which has x,y
  and z components.
\item
  Jerk signals are calculated for BodyAcceleration readings.
\item
  Fourier Transforms are made on the above time readings to obtain
  frequency readings.
\item
  Now, on all the base signal readings., mean, max, mad, sma,
  arcoefficient, engerybands,entropy etc., are calculated for each
  window.
\item
  We get a feature vector of 561 features and these features are given
  in the dataset.
\item
  Each window of readings is a datapoint of 561 features.
\end{itemize}

\hypertarget{problem-framework}{%
\subsection{Problem Framework}\label{problem-framework}}

\begin{itemize}
\tightlist
\item
  30 subjects(volunteers) data is randomly split to 70\%(21) test and
  30\%(7) train data.
\item
  Each datapoint corresponds one of the 6 Activities.
\end{itemize}

    \hypertarget{problem-statement}{%
\subsection{Problem Statement}\label{problem-statement}}

\begin{itemize}
\tightlist
\item
  Given a new datapoint we have to predict the Activity
\end{itemize}

    

    \begin{Verbatim}[commandchars=\\\{\}]
{\color{incolor}In [{\color{incolor}1}]:} \PY{k+kn}{import} \PY{n+nn}{numpy} \PY{k}{as} \PY{n+nn}{np}
        \PY{k+kn}{import} \PY{n+nn}{pandas} \PY{k}{as} \PY{n+nn}{pd}
        
        \PY{c+c1}{\PYZsh{} get the features from the file features.txt}
        \PY{n}{features} \PY{o}{=} \PY{n+nb}{list}\PY{p}{(}\PY{p}{)}
        \PY{k}{with} \PY{n+nb}{open}\PY{p}{(}\PY{l+s+s1}{\PYZsq{}}\PY{l+s+s1}{UCI\PYZus{}HAR\PYZus{}Dataset/features.txt}\PY{l+s+s1}{\PYZsq{}}\PY{p}{)} \PY{k}{as} \PY{n}{f}\PY{p}{:}
            \PY{n}{features} \PY{o}{=} \PY{p}{[}\PY{n}{line}\PY{o}{.}\PY{n}{split}\PY{p}{(}\PY{p}{)}\PY{p}{[}\PY{l+m+mi}{1}\PY{p}{]} \PY{k}{for} \PY{n}{line} \PY{o+ow}{in} \PY{n}{f}\PY{o}{.}\PY{n}{readlines}\PY{p}{(}\PY{p}{)}\PY{p}{]}
        \PY{n+nb}{print}\PY{p}{(}\PY{l+s+s1}{\PYZsq{}}\PY{l+s+s1}{No of Features: }\PY{l+s+si}{\PYZob{}\PYZcb{}}\PY{l+s+s1}{\PYZsq{}}\PY{o}{.}\PY{n}{format}\PY{p}{(}\PY{n+nb}{len}\PY{p}{(}\PY{n}{features}\PY{p}{)}\PY{p}{)}\PY{p}{)}
\end{Verbatim}


    \begin{Verbatim}[commandchars=\\\{\}]
No of Features: 561

    \end{Verbatim}

    \hypertarget{obtain-the-train-data}{%
\subsection{Obtain the train data}\label{obtain-the-train-data}}

    \begin{Verbatim}[commandchars=\\\{\}]
{\color{incolor}In [{\color{incolor}2}]:} \PY{c+c1}{\PYZsh{} get the data from txt files to pandas dataffame}
        \PY{n}{X\PYZus{}train} \PY{o}{=} \PY{n}{pd}\PY{o}{.}\PY{n}{read\PYZus{}csv}\PY{p}{(}\PY{l+s+s1}{\PYZsq{}}\PY{l+s+s1}{UCI\PYZus{}HAR\PYZus{}dataset/train/X\PYZus{}train.txt}\PY{l+s+s1}{\PYZsq{}}\PY{p}{,} \PY{n}{delim\PYZus{}whitespace}\PY{o}{=}\PY{k+kc}{True}\PY{p}{,} \PY{n}{header}\PY{o}{=}\PY{k+kc}{None}\PY{p}{,} \PY{n}{names}\PY{o}{=}\PY{n}{features}\PY{p}{)}
        
        \PY{c+c1}{\PYZsh{} add subject column to the dataframe}
        \PY{n}{X\PYZus{}train}\PY{p}{[}\PY{l+s+s1}{\PYZsq{}}\PY{l+s+s1}{subject}\PY{l+s+s1}{\PYZsq{}}\PY{p}{]} \PY{o}{=} \PY{n}{pd}\PY{o}{.}\PY{n}{read\PYZus{}csv}\PY{p}{(}\PY{l+s+s1}{\PYZsq{}}\PY{l+s+s1}{UCI\PYZus{}HAR\PYZus{}dataset/train/subject\PYZus{}train.txt}\PY{l+s+s1}{\PYZsq{}}\PY{p}{,} \PY{n}{header}\PY{o}{=}\PY{k+kc}{None}\PY{p}{,} \PY{n}{squeeze}\PY{o}{=}\PY{k+kc}{True}\PY{p}{)}
        
        \PY{n}{y\PYZus{}train} \PY{o}{=} \PY{n}{pd}\PY{o}{.}\PY{n}{read\PYZus{}csv}\PY{p}{(}\PY{l+s+s1}{\PYZsq{}}\PY{l+s+s1}{UCI\PYZus{}HAR\PYZus{}dataset/train/y\PYZus{}train.txt}\PY{l+s+s1}{\PYZsq{}}\PY{p}{,} \PY{n}{names}\PY{o}{=}\PY{p}{[}\PY{l+s+s1}{\PYZsq{}}\PY{l+s+s1}{Activity}\PY{l+s+s1}{\PYZsq{}}\PY{p}{]}\PY{p}{,} \PY{n}{squeeze}\PY{o}{=}\PY{k+kc}{True}\PY{p}{)}
        \PY{n}{y\PYZus{}train\PYZus{}labels} \PY{o}{=} \PY{n}{y\PYZus{}train}\PY{o}{.}\PY{n}{map}\PY{p}{(}\PY{p}{\PYZob{}}\PY{l+m+mi}{1}\PY{p}{:} \PY{l+s+s1}{\PYZsq{}}\PY{l+s+s1}{WALKING}\PY{l+s+s1}{\PYZsq{}}\PY{p}{,} \PY{l+m+mi}{2}\PY{p}{:}\PY{l+s+s1}{\PYZsq{}}\PY{l+s+s1}{WALKING\PYZus{}UPSTAIRS}\PY{l+s+s1}{\PYZsq{}}\PY{p}{,}\PY{l+m+mi}{3}\PY{p}{:}\PY{l+s+s1}{\PYZsq{}}\PY{l+s+s1}{WALKING\PYZus{}DOWNSTAIRS}\PY{l+s+s1}{\PYZsq{}}\PY{p}{,}\PYZbs{}
                               \PY{l+m+mi}{4}\PY{p}{:}\PY{l+s+s1}{\PYZsq{}}\PY{l+s+s1}{SITTING}\PY{l+s+s1}{\PYZsq{}}\PY{p}{,} \PY{l+m+mi}{5}\PY{p}{:}\PY{l+s+s1}{\PYZsq{}}\PY{l+s+s1}{STANDING}\PY{l+s+s1}{\PYZsq{}}\PY{p}{,}\PY{l+m+mi}{6}\PY{p}{:}\PY{l+s+s1}{\PYZsq{}}\PY{l+s+s1}{LAYING}\PY{l+s+s1}{\PYZsq{}}\PY{p}{\PYZcb{}}\PY{p}{)}
        
        \PY{c+c1}{\PYZsh{} put all columns in a single dataframe}
        \PY{n}{train} \PY{o}{=} \PY{n}{X\PYZus{}train}
        \PY{n}{train}\PY{p}{[}\PY{l+s+s1}{\PYZsq{}}\PY{l+s+s1}{Activity}\PY{l+s+s1}{\PYZsq{}}\PY{p}{]} \PY{o}{=} \PY{n}{y\PYZus{}train}
        \PY{n}{train}\PY{p}{[}\PY{l+s+s1}{\PYZsq{}}\PY{l+s+s1}{ActivityName}\PY{l+s+s1}{\PYZsq{}}\PY{p}{]} \PY{o}{=} \PY{n}{y\PYZus{}train\PYZus{}labels}
        \PY{n}{train}\PY{o}{.}\PY{n}{sample}\PY{p}{(}\PY{p}{)}
\end{Verbatim}


    \begin{Verbatim}[commandchars=\\\{\}]
D:\textbackslash{}installed\textbackslash{}Anaconda3\textbackslash{}lib\textbackslash{}site-packages\textbackslash{}pandas\textbackslash{}io\textbackslash{}parsers.py:678: UserWarning: Duplicate names specified. This will raise an error in the future.
  return \_read(filepath\_or\_buffer, kwds)

    \end{Verbatim}

\begin{Verbatim}[commandchars=\\\{\}]
{\color{outcolor}Out[{\color{outcolor}2}]:}       tBodyAcc-mean()-X  tBodyAcc-mean()-Y  tBodyAcc-mean()-Z  \textbackslash{}
        6015             0.2797          -0.004397           -0.10952   
        
              tBodyAcc-std()-X  tBodyAcc-std()-Y  tBodyAcc-std()-Z  tBodyAcc-mad()-X  \textbackslash{}
        6015          0.359081          0.119909         -0.177541          0.337963   
        
              tBodyAcc-mad()-Y  tBodyAcc-mad()-Z  tBodyAcc-max()-X  \textbackslash{}
        6015          0.066883         -0.221876          0.474093   
        
                     {\ldots}          angle(tBodyAccMean,gravity)  \textbackslash{}
        6015         {\ldots}                             0.049658   
        
              angle(tBodyAccJerkMean),gravityMean)  angle(tBodyGyroMean,gravityMean)  \textbackslash{}
        6015                              0.602595                          0.681696   
        
              angle(tBodyGyroJerkMean,gravityMean)  angle(X,gravityMean)  \textbackslash{}
        6015                               0.51333             -0.862824   
        
              angle(Y,gravityMean)  angle(Z,gravityMean)  subject  Activity  \textbackslash{}
        6015              0.190833              0.038933       27         3   
        
                    ActivityName  
        6015  WALKING\_DOWNSTAIRS  
        
        [1 rows x 564 columns]
\end{Verbatim}
            
    \begin{Verbatim}[commandchars=\\\{\}]
{\color{incolor}In [{\color{incolor}3}]:} \PY{n}{train}\PY{o}{.}\PY{n}{shape}
\end{Verbatim}


\begin{Verbatim}[commandchars=\\\{\}]
{\color{outcolor}Out[{\color{outcolor}3}]:} (7352, 564)
\end{Verbatim}
            
    \hypertarget{obtain-the-test-data}{%
\subsection{Obtain the test data}\label{obtain-the-test-data}}

    \begin{Verbatim}[commandchars=\\\{\}]
{\color{incolor}In [{\color{incolor}4}]:} \PY{c+c1}{\PYZsh{} get the data from txt files to pandas dataffame}
        \PY{n}{X\PYZus{}test} \PY{o}{=} \PY{n}{pd}\PY{o}{.}\PY{n}{read\PYZus{}csv}\PY{p}{(}\PY{l+s+s1}{\PYZsq{}}\PY{l+s+s1}{UCI\PYZus{}HAR\PYZus{}dataset/test/X\PYZus{}test.txt}\PY{l+s+s1}{\PYZsq{}}\PY{p}{,} \PY{n}{delim\PYZus{}whitespace}\PY{o}{=}\PY{k+kc}{True}\PY{p}{,} \PY{n}{header}\PY{o}{=}\PY{k+kc}{None}\PY{p}{,} \PY{n}{names}\PY{o}{=}\PY{n}{features}\PY{p}{)}
        
        \PY{c+c1}{\PYZsh{} add subject column to the dataframe}
        \PY{n}{X\PYZus{}test}\PY{p}{[}\PY{l+s+s1}{\PYZsq{}}\PY{l+s+s1}{subject}\PY{l+s+s1}{\PYZsq{}}\PY{p}{]} \PY{o}{=} \PY{n}{pd}\PY{o}{.}\PY{n}{read\PYZus{}csv}\PY{p}{(}\PY{l+s+s1}{\PYZsq{}}\PY{l+s+s1}{UCI\PYZus{}HAR\PYZus{}dataset/test/subject\PYZus{}test.txt}\PY{l+s+s1}{\PYZsq{}}\PY{p}{,} \PY{n}{header}\PY{o}{=}\PY{k+kc}{None}\PY{p}{,} \PY{n}{squeeze}\PY{o}{=}\PY{k+kc}{True}\PY{p}{)}
        
        \PY{c+c1}{\PYZsh{} get y labels from the txt file}
        \PY{n}{y\PYZus{}test} \PY{o}{=} \PY{n}{pd}\PY{o}{.}\PY{n}{read\PYZus{}csv}\PY{p}{(}\PY{l+s+s1}{\PYZsq{}}\PY{l+s+s1}{UCI\PYZus{}HAR\PYZus{}dataset/test/y\PYZus{}test.txt}\PY{l+s+s1}{\PYZsq{}}\PY{p}{,} \PY{n}{names}\PY{o}{=}\PY{p}{[}\PY{l+s+s1}{\PYZsq{}}\PY{l+s+s1}{Activity}\PY{l+s+s1}{\PYZsq{}}\PY{p}{]}\PY{p}{,} \PY{n}{squeeze}\PY{o}{=}\PY{k+kc}{True}\PY{p}{)}
        \PY{n}{y\PYZus{}test\PYZus{}labels} \PY{o}{=} \PY{n}{y\PYZus{}test}\PY{o}{.}\PY{n}{map}\PY{p}{(}\PY{p}{\PYZob{}}\PY{l+m+mi}{1}\PY{p}{:} \PY{l+s+s1}{\PYZsq{}}\PY{l+s+s1}{WALKING}\PY{l+s+s1}{\PYZsq{}}\PY{p}{,} \PY{l+m+mi}{2}\PY{p}{:}\PY{l+s+s1}{\PYZsq{}}\PY{l+s+s1}{WALKING\PYZus{}UPSTAIRS}\PY{l+s+s1}{\PYZsq{}}\PY{p}{,}\PY{l+m+mi}{3}\PY{p}{:}\PY{l+s+s1}{\PYZsq{}}\PY{l+s+s1}{WALKING\PYZus{}DOWNSTAIRS}\PY{l+s+s1}{\PYZsq{}}\PY{p}{,}\PYZbs{}
                               \PY{l+m+mi}{4}\PY{p}{:}\PY{l+s+s1}{\PYZsq{}}\PY{l+s+s1}{SITTING}\PY{l+s+s1}{\PYZsq{}}\PY{p}{,} \PY{l+m+mi}{5}\PY{p}{:}\PY{l+s+s1}{\PYZsq{}}\PY{l+s+s1}{STANDING}\PY{l+s+s1}{\PYZsq{}}\PY{p}{,}\PY{l+m+mi}{6}\PY{p}{:}\PY{l+s+s1}{\PYZsq{}}\PY{l+s+s1}{LAYING}\PY{l+s+s1}{\PYZsq{}}\PY{p}{\PYZcb{}}\PY{p}{)}
        
        
        \PY{c+c1}{\PYZsh{} put all columns in a single dataframe}
        \PY{n}{test} \PY{o}{=} \PY{n}{X\PYZus{}test}
        \PY{n}{test}\PY{p}{[}\PY{l+s+s1}{\PYZsq{}}\PY{l+s+s1}{Activity}\PY{l+s+s1}{\PYZsq{}}\PY{p}{]} \PY{o}{=} \PY{n}{y\PYZus{}test}
        \PY{n}{test}\PY{p}{[}\PY{l+s+s1}{\PYZsq{}}\PY{l+s+s1}{ActivityName}\PY{l+s+s1}{\PYZsq{}}\PY{p}{]} \PY{o}{=} \PY{n}{y\PYZus{}test\PYZus{}labels}
        \PY{n}{test}\PY{o}{.}\PY{n}{sample}\PY{p}{(}\PY{p}{)}
\end{Verbatim}


    \begin{Verbatim}[commandchars=\\\{\}]
D:\textbackslash{}installed\textbackslash{}Anaconda3\textbackslash{}lib\textbackslash{}site-packages\textbackslash{}pandas\textbackslash{}io\textbackslash{}parsers.py:678: UserWarning: Duplicate names specified. This will raise an error in the future.
  return \_read(filepath\_or\_buffer, kwds)

    \end{Verbatim}

\begin{Verbatim}[commandchars=\\\{\}]
{\color{outcolor}Out[{\color{outcolor}4}]:}       tBodyAcc-mean()-X  tBodyAcc-mean()-Y  tBodyAcc-mean()-Z  \textbackslash{}
        2261           0.279196          -0.018261          -0.103376   
        
              tBodyAcc-std()-X  tBodyAcc-std()-Y  tBodyAcc-std()-Z  tBodyAcc-mad()-X  \textbackslash{}
        2261         -0.996955         -0.982959         -0.988239           -0.9972   
        
              tBodyAcc-mad()-Y  tBodyAcc-mad()-Z  tBodyAcc-max()-X      {\ldots}       \textbackslash{}
        2261         -0.982509         -0.986964         -0.940634      {\ldots}        
        
              angle(tBodyAccMean,gravity)  angle(tBodyAccJerkMean),gravityMean)  \textbackslash{}
        2261                    -0.268441                             -0.215632   
        
              angle(tBodyGyroMean,gravityMean)  angle(tBodyGyroJerkMean,gravityMean)  \textbackslash{}
        2261                         -0.465366                              0.098119   
        
              angle(X,gravityMean)  angle(Y,gravityMean)  angle(Z,gravityMean)  \textbackslash{}
        2261             -0.612458             -0.033918             -0.224544   
        
              subject  Activity  ActivityName  
        2261       20         4       SITTING  
        
        [1 rows x 564 columns]
\end{Verbatim}
            
    \begin{Verbatim}[commandchars=\\\{\}]
{\color{incolor}In [{\color{incolor}5}]:} \PY{n}{test}\PY{o}{.}\PY{n}{shape}
\end{Verbatim}


\begin{Verbatim}[commandchars=\\\{\}]
{\color{outcolor}Out[{\color{outcolor}5}]:} (2947, 564)
\end{Verbatim}
            
    

    \hypertarget{data-cleaning}{%
\section{Data Cleaning}\label{data-cleaning}}

    \hypertarget{check-for-duplicates}{%
\subsection{1. Check for Duplicates}\label{check-for-duplicates}}

    \begin{Verbatim}[commandchars=\\\{\}]
{\color{incolor}In [{\color{incolor}6}]:} \PY{n+nb}{print}\PY{p}{(}\PY{l+s+s1}{\PYZsq{}}\PY{l+s+s1}{No of duplicates in train: }\PY{l+s+si}{\PYZob{}\PYZcb{}}\PY{l+s+s1}{\PYZsq{}}\PY{o}{.}\PY{n}{format}\PY{p}{(}\PY{n+nb}{sum}\PY{p}{(}\PY{n}{train}\PY{o}{.}\PY{n}{duplicated}\PY{p}{(}\PY{p}{)}\PY{p}{)}\PY{p}{)}\PY{p}{)}
        \PY{n+nb}{print}\PY{p}{(}\PY{l+s+s1}{\PYZsq{}}\PY{l+s+s1}{No of duplicates in test : }\PY{l+s+si}{\PYZob{}\PYZcb{}}\PY{l+s+s1}{\PYZsq{}}\PY{o}{.}\PY{n}{format}\PY{p}{(}\PY{n+nb}{sum}\PY{p}{(}\PY{n}{test}\PY{o}{.}\PY{n}{duplicated}\PY{p}{(}\PY{p}{)}\PY{p}{)}\PY{p}{)}\PY{p}{)}
\end{Verbatim}


    \begin{Verbatim}[commandchars=\\\{\}]
No of duplicates in train: 0
No of duplicates in test : 0

    \end{Verbatim}

    

    \hypertarget{checking-for-nannull-values}{%
\subsection{2. Checking for NaN/null
values}\label{checking-for-nannull-values}}

    \begin{Verbatim}[commandchars=\\\{\}]
{\color{incolor}In [{\color{incolor}7}]:} \PY{n+nb}{print}\PY{p}{(}\PY{l+s+s1}{\PYZsq{}}\PY{l+s+s1}{We have }\PY{l+s+si}{\PYZob{}\PYZcb{}}\PY{l+s+s1}{ NaN/Null values in train}\PY{l+s+s1}{\PYZsq{}}\PY{o}{.}\PY{n}{format}\PY{p}{(}\PY{n}{train}\PY{o}{.}\PY{n}{isnull}\PY{p}{(}\PY{p}{)}\PY{o}{.}\PY{n}{values}\PY{o}{.}\PY{n}{sum}\PY{p}{(}\PY{p}{)}\PY{p}{)}\PY{p}{)}
        \PY{n+nb}{print}\PY{p}{(}\PY{l+s+s1}{\PYZsq{}}\PY{l+s+s1}{We have }\PY{l+s+si}{\PYZob{}\PYZcb{}}\PY{l+s+s1}{ NaN/Null values in test}\PY{l+s+s1}{\PYZsq{}}\PY{o}{.}\PY{n}{format}\PY{p}{(}\PY{n}{test}\PY{o}{.}\PY{n}{isnull}\PY{p}{(}\PY{p}{)}\PY{o}{.}\PY{n}{values}\PY{o}{.}\PY{n}{sum}\PY{p}{(}\PY{p}{)}\PY{p}{)}\PY{p}{)}
\end{Verbatim}


    \begin{Verbatim}[commandchars=\\\{\}]
We have 0 NaN/Null values in train
We have 0 NaN/Null values in test

    \end{Verbatim}

    

    

    \hypertarget{check-for-data-imbalance}{%
\subsection{3. Check for data
imbalance}\label{check-for-data-imbalance}}

    \begin{Verbatim}[commandchars=\\\{\}]
{\color{incolor}In [{\color{incolor}8}]:} \PY{k+kn}{import} \PY{n+nn}{matplotlib}\PY{n+nn}{.}\PY{n+nn}{pyplot} \PY{k}{as} \PY{n+nn}{plt}
        \PY{k+kn}{import} \PY{n+nn}{seaborn} \PY{k}{as} \PY{n+nn}{sns}
        
        \PY{n}{sns}\PY{o}{.}\PY{n}{set\PYZus{}style}\PY{p}{(}\PY{l+s+s1}{\PYZsq{}}\PY{l+s+s1}{whitegrid}\PY{l+s+s1}{\PYZsq{}}\PY{p}{)}
        \PY{n}{plt}\PY{o}{.}\PY{n}{rcParams}\PY{p}{[}\PY{l+s+s1}{\PYZsq{}}\PY{l+s+s1}{font.family}\PY{l+s+s1}{\PYZsq{}}\PY{p}{]} \PY{o}{=} \PY{l+s+s1}{\PYZsq{}}\PY{l+s+s1}{Dejavu Sans}\PY{l+s+s1}{\PYZsq{}}
\end{Verbatim}


    \begin{Verbatim}[commandchars=\\\{\}]
{\color{incolor}In [{\color{incolor}9}]:} \PY{n}{plt}\PY{o}{.}\PY{n}{figure}\PY{p}{(}\PY{n}{figsize}\PY{o}{=}\PY{p}{(}\PY{l+m+mi}{16}\PY{p}{,}\PY{l+m+mi}{8}\PY{p}{)}\PY{p}{)}
        \PY{n}{plt}\PY{o}{.}\PY{n}{title}\PY{p}{(}\PY{l+s+s1}{\PYZsq{}}\PY{l+s+s1}{Data provided by each user}\PY{l+s+s1}{\PYZsq{}}\PY{p}{,} \PY{n}{fontsize}\PY{o}{=}\PY{l+m+mi}{20}\PY{p}{)}
        \PY{n}{sns}\PY{o}{.}\PY{n}{countplot}\PY{p}{(}\PY{n}{x}\PY{o}{=}\PY{l+s+s1}{\PYZsq{}}\PY{l+s+s1}{subject}\PY{l+s+s1}{\PYZsq{}}\PY{p}{,}\PY{n}{hue}\PY{o}{=}\PY{l+s+s1}{\PYZsq{}}\PY{l+s+s1}{ActivityName}\PY{l+s+s1}{\PYZsq{}}\PY{p}{,} \PY{n}{data} \PY{o}{=} \PY{n}{train}\PY{p}{)}
        \PY{n}{plt}\PY{o}{.}\PY{n}{show}\PY{p}{(}\PY{p}{)}
\end{Verbatim}


    \begin{center}
    \adjustimage{max size={0.9\linewidth}{0.9\paperheight}}{output_26_0.png}
    \end{center}
    { \hspace*{\fill} \\}
    
    \begin{quote}
We have got almost same number of reading from all the subjects
\end{quote}

    \begin{Verbatim}[commandchars=\\\{\}]
{\color{incolor}In [{\color{incolor}10}]:} \PY{n}{plt}\PY{o}{.}\PY{n}{title}\PY{p}{(}\PY{l+s+s1}{\PYZsq{}}\PY{l+s+s1}{No of Datapoints per Activity}\PY{l+s+s1}{\PYZsq{}}\PY{p}{,} \PY{n}{fontsize}\PY{o}{=}\PY{l+m+mi}{15}\PY{p}{)}
         \PY{n}{sns}\PY{o}{.}\PY{n}{countplot}\PY{p}{(}\PY{n}{train}\PY{o}{.}\PY{n}{ActivityName}\PY{p}{)}
         \PY{n}{plt}\PY{o}{.}\PY{n}{xticks}\PY{p}{(}\PY{n}{rotation}\PY{o}{=}\PY{l+m+mi}{90}\PY{p}{)}
         \PY{n}{plt}\PY{o}{.}\PY{n}{show}\PY{p}{(}\PY{p}{)}
\end{Verbatim}


    \begin{center}
    \adjustimage{max size={0.9\linewidth}{0.9\paperheight}}{output_28_0.png}
    \end{center}
    { \hspace*{\fill} \\}
    
    \hypertarget{observation}{%
\subsubsection{Observation}\label{observation}}

\begin{quote}
Our data is well balanced (almost)
\end{quote}

    \hypertarget{changing-feature-names}{%
\subsection{4. Changing feature names}\label{changing-feature-names}}

    \begin{Verbatim}[commandchars=\\\{\}]
{\color{incolor}In [{\color{incolor}11}]:} \PY{n}{columns} \PY{o}{=} \PY{n}{train}\PY{o}{.}\PY{n}{columns}
         
         \PY{c+c1}{\PYZsh{} Removing \PYZsq{}()\PYZsq{} from column names}
         \PY{n}{columns} \PY{o}{=} \PY{n}{columns}\PY{o}{.}\PY{n}{str}\PY{o}{.}\PY{n}{replace}\PY{p}{(}\PY{l+s+s1}{\PYZsq{}}\PY{l+s+s1}{[()]}\PY{l+s+s1}{\PYZsq{}}\PY{p}{,}\PY{l+s+s1}{\PYZsq{}}\PY{l+s+s1}{\PYZsq{}}\PY{p}{)}
         \PY{n}{columns} \PY{o}{=} \PY{n}{columns}\PY{o}{.}\PY{n}{str}\PY{o}{.}\PY{n}{replace}\PY{p}{(}\PY{l+s+s1}{\PYZsq{}}\PY{l+s+s1}{[\PYZhy{}]}\PY{l+s+s1}{\PYZsq{}}\PY{p}{,} \PY{l+s+s1}{\PYZsq{}}\PY{l+s+s1}{\PYZsq{}}\PY{p}{)}
         \PY{n}{columns} \PY{o}{=} \PY{n}{columns}\PY{o}{.}\PY{n}{str}\PY{o}{.}\PY{n}{replace}\PY{p}{(}\PY{l+s+s1}{\PYZsq{}}\PY{l+s+s1}{[,]}\PY{l+s+s1}{\PYZsq{}}\PY{p}{,}\PY{l+s+s1}{\PYZsq{}}\PY{l+s+s1}{\PYZsq{}}\PY{p}{)}
         
         \PY{n}{train}\PY{o}{.}\PY{n}{columns} \PY{o}{=} \PY{n}{columns}
         \PY{n}{test}\PY{o}{.}\PY{n}{columns} \PY{o}{=} \PY{n}{columns}
         
         \PY{n}{test}\PY{o}{.}\PY{n}{columns}
\end{Verbatim}


\begin{Verbatim}[commandchars=\\\{\}]
{\color{outcolor}Out[{\color{outcolor}11}]:} Index(['tBodyAccmeanX', 'tBodyAccmeanY', 'tBodyAccmeanZ', 'tBodyAccstdX',
                'tBodyAccstdY', 'tBodyAccstdZ', 'tBodyAccmadX', 'tBodyAccmadY',
                'tBodyAccmadZ', 'tBodyAccmaxX',
                {\ldots}
                'angletBodyAccMeangravity', 'angletBodyAccJerkMeangravityMean',
                'angletBodyGyroMeangravityMean', 'angletBodyGyroJerkMeangravityMean',
                'angleXgravityMean', 'angleYgravityMean', 'angleZgravityMean',
                'subject', 'Activity', 'ActivityName'],
               dtype='object', length=564)
\end{Verbatim}
            
    \hypertarget{save-this-dataframe-in-a-csv-files}{%
\subsection{5. Save this dataframe in a csv
files}\label{save-this-dataframe-in-a-csv-files}}

    \begin{Verbatim}[commandchars=\\\{\}]
{\color{incolor}In [{\color{incolor}13}]:} \PY{n}{train}\PY{o}{.}\PY{n}{to\PYZus{}csv}\PY{p}{(}\PY{l+s+s1}{\PYZsq{}}\PY{l+s+s1}{UCI\PYZus{}HAR\PYZus{}Dataset/csv\PYZus{}files/train.csv}\PY{l+s+s1}{\PYZsq{}}\PY{p}{,} \PY{n}{index}\PY{o}{=}\PY{k+kc}{False}\PY{p}{)}
         \PY{n}{test}\PY{o}{.}\PY{n}{to\PYZus{}csv}\PY{p}{(}\PY{l+s+s1}{\PYZsq{}}\PY{l+s+s1}{UCI\PYZus{}HAR\PYZus{}Dataset/csv\PYZus{}files/test.csv}\PY{l+s+s1}{\PYZsq{}}\PY{p}{,} \PY{n}{index}\PY{o}{=}\PY{k+kc}{False}\PY{p}{)}
\end{Verbatim}


    

    \hypertarget{exploratory-data-analysis}{%
\section{Exploratory Data Analysis}\label{exploratory-data-analysis}}

    ``\textbf{\emph{Without domain knowledge EDA has no meaning, without EDA
a problem has no soul.}}''

    \hypertarget{featuring-engineering-from-domain-knowledge}{%
\subsubsection{1. Featuring Engineering from Domain
Knowledge}\label{featuring-engineering-from-domain-knowledge}}

    \begin{itemize}
\item
  \textbf{Static and Dynamic Activities}

  \begin{itemize}
  \tightlist
  \item
    In static activities (sit, stand, lie down) motion information will
    not be very useful.
  \item
    In the dynamic activities (Walking,
    WalkingUpstairs,WalkingDownstairs) motion info will be significant.
  \end{itemize}
\end{itemize}

    \hypertarget{stationary-and-moving-activities-are-completely-different}{%
\subsubsection{2. Stationary and Moving activities are completely
different}\label{stationary-and-moving-activities-are-completely-different}}

    \begin{Verbatim}[commandchars=\\\{\}]
{\color{incolor}In [{\color{incolor}14}]:} \PY{n}{sns}\PY{o}{.}\PY{n}{set\PYZus{}palette}\PY{p}{(}\PY{l+s+s2}{\PYZdq{}}\PY{l+s+s2}{Set1}\PY{l+s+s2}{\PYZdq{}}\PY{p}{,} \PY{n}{desat}\PY{o}{=}\PY{l+m+mf}{0.80}\PY{p}{)}
         \PY{n}{facetgrid} \PY{o}{=} \PY{n}{sns}\PY{o}{.}\PY{n}{FacetGrid}\PY{p}{(}\PY{n}{train}\PY{p}{,} \PY{n}{hue}\PY{o}{=}\PY{l+s+s1}{\PYZsq{}}\PY{l+s+s1}{ActivityName}\PY{l+s+s1}{\PYZsq{}}\PY{p}{,} \PY{n}{size}\PY{o}{=}\PY{l+m+mi}{6}\PY{p}{,}\PY{n}{aspect}\PY{o}{=}\PY{l+m+mi}{2}\PY{p}{)}
         \PY{n}{facetgrid}\PY{o}{.}\PY{n}{map}\PY{p}{(}\PY{n}{sns}\PY{o}{.}\PY{n}{distplot}\PY{p}{,}\PY{l+s+s1}{\PYZsq{}}\PY{l+s+s1}{tBodyAccMagmean}\PY{l+s+s1}{\PYZsq{}}\PY{p}{,} \PY{n}{hist}\PY{o}{=}\PY{k+kc}{False}\PY{p}{)}\PYZbs{}
             \PY{o}{.}\PY{n}{add\PYZus{}legend}\PY{p}{(}\PY{p}{)}
         \PY{n}{plt}\PY{o}{.}\PY{n}{annotate}\PY{p}{(}\PY{l+s+s2}{\PYZdq{}}\PY{l+s+s2}{Stationary Activities}\PY{l+s+s2}{\PYZdq{}}\PY{p}{,} \PY{n}{xy}\PY{o}{=}\PY{p}{(}\PY{o}{\PYZhy{}}\PY{l+m+mf}{0.956}\PY{p}{,}\PY{l+m+mi}{17}\PY{p}{)}\PY{p}{,} \PY{n}{xytext}\PY{o}{=}\PY{p}{(}\PY{o}{\PYZhy{}}\PY{l+m+mf}{0.9}\PY{p}{,} \PY{l+m+mi}{23}\PY{p}{)}\PY{p}{,} \PY{n}{size}\PY{o}{=}\PY{l+m+mi}{20}\PY{p}{,}\PYZbs{}
                     \PY{n}{va}\PY{o}{=}\PY{l+s+s1}{\PYZsq{}}\PY{l+s+s1}{center}\PY{l+s+s1}{\PYZsq{}}\PY{p}{,} \PY{n}{ha}\PY{o}{=}\PY{l+s+s1}{\PYZsq{}}\PY{l+s+s1}{left}\PY{l+s+s1}{\PYZsq{}}\PY{p}{,}\PYZbs{}
                     \PY{n}{arrowprops}\PY{o}{=}\PY{n+nb}{dict}\PY{p}{(}\PY{n}{arrowstyle}\PY{o}{=}\PY{l+s+s2}{\PYZdq{}}\PY{l+s+s2}{simple}\PY{l+s+s2}{\PYZdq{}}\PY{p}{,}\PY{n}{connectionstyle}\PY{o}{=}\PY{l+s+s2}{\PYZdq{}}\PY{l+s+s2}{arc3,rad=0.1}\PY{l+s+s2}{\PYZdq{}}\PY{p}{)}\PY{p}{)}
         
         \PY{n}{plt}\PY{o}{.}\PY{n}{annotate}\PY{p}{(}\PY{l+s+s2}{\PYZdq{}}\PY{l+s+s2}{Moving Activities}\PY{l+s+s2}{\PYZdq{}}\PY{p}{,} \PY{n}{xy}\PY{o}{=}\PY{p}{(}\PY{l+m+mi}{0}\PY{p}{,}\PY{l+m+mi}{3}\PY{p}{)}\PY{p}{,} \PY{n}{xytext}\PY{o}{=}\PY{p}{(}\PY{l+m+mf}{0.2}\PY{p}{,} \PY{l+m+mi}{9}\PY{p}{)}\PY{p}{,} \PY{n}{size}\PY{o}{=}\PY{l+m+mi}{20}\PY{p}{,}\PYZbs{}
                     \PY{n}{va}\PY{o}{=}\PY{l+s+s1}{\PYZsq{}}\PY{l+s+s1}{center}\PY{l+s+s1}{\PYZsq{}}\PY{p}{,} \PY{n}{ha}\PY{o}{=}\PY{l+s+s1}{\PYZsq{}}\PY{l+s+s1}{left}\PY{l+s+s1}{\PYZsq{}}\PY{p}{,}\PYZbs{}
                     \PY{n}{arrowprops}\PY{o}{=}\PY{n+nb}{dict}\PY{p}{(}\PY{n}{arrowstyle}\PY{o}{=}\PY{l+s+s2}{\PYZdq{}}\PY{l+s+s2}{simple}\PY{l+s+s2}{\PYZdq{}}\PY{p}{,}\PY{n}{connectionstyle}\PY{o}{=}\PY{l+s+s2}{\PYZdq{}}\PY{l+s+s2}{arc3,rad=0.1}\PY{l+s+s2}{\PYZdq{}}\PY{p}{)}\PY{p}{)}
         \PY{n}{plt}\PY{o}{.}\PY{n}{show}\PY{p}{(}\PY{p}{)}
\end{Verbatim}


    \begin{center}
    \adjustimage{max size={0.9\linewidth}{0.9\paperheight}}{output_40_0.png}
    \end{center}
    { \hspace*{\fill} \\}
    
    \begin{Verbatim}[commandchars=\\\{\}]
{\color{incolor}In [{\color{incolor}15}]:} \PY{c+c1}{\PYZsh{} for plotting purposes taking datapoints of each activity to a different dataframe}
         \PY{n}{df1} \PY{o}{=} \PY{n}{train}\PY{p}{[}\PY{n}{train}\PY{p}{[}\PY{l+s+s1}{\PYZsq{}}\PY{l+s+s1}{Activity}\PY{l+s+s1}{\PYZsq{}}\PY{p}{]}\PY{o}{==}\PY{l+m+mi}{1}\PY{p}{]}
         \PY{n}{df2} \PY{o}{=} \PY{n}{train}\PY{p}{[}\PY{n}{train}\PY{p}{[}\PY{l+s+s1}{\PYZsq{}}\PY{l+s+s1}{Activity}\PY{l+s+s1}{\PYZsq{}}\PY{p}{]}\PY{o}{==}\PY{l+m+mi}{2}\PY{p}{]}
         \PY{n}{df3} \PY{o}{=} \PY{n}{train}\PY{p}{[}\PY{n}{train}\PY{p}{[}\PY{l+s+s1}{\PYZsq{}}\PY{l+s+s1}{Activity}\PY{l+s+s1}{\PYZsq{}}\PY{p}{]}\PY{o}{==}\PY{l+m+mi}{3}\PY{p}{]}
         \PY{n}{df4} \PY{o}{=} \PY{n}{train}\PY{p}{[}\PY{n}{train}\PY{p}{[}\PY{l+s+s1}{\PYZsq{}}\PY{l+s+s1}{Activity}\PY{l+s+s1}{\PYZsq{}}\PY{p}{]}\PY{o}{==}\PY{l+m+mi}{4}\PY{p}{]}
         \PY{n}{df5} \PY{o}{=} \PY{n}{train}\PY{p}{[}\PY{n}{train}\PY{p}{[}\PY{l+s+s1}{\PYZsq{}}\PY{l+s+s1}{Activity}\PY{l+s+s1}{\PYZsq{}}\PY{p}{]}\PY{o}{==}\PY{l+m+mi}{5}\PY{p}{]}
         \PY{n}{df6} \PY{o}{=} \PY{n}{train}\PY{p}{[}\PY{n}{train}\PY{p}{[}\PY{l+s+s1}{\PYZsq{}}\PY{l+s+s1}{Activity}\PY{l+s+s1}{\PYZsq{}}\PY{p}{]}\PY{o}{==}\PY{l+m+mi}{6}\PY{p}{]}
         
         \PY{n}{plt}\PY{o}{.}\PY{n}{figure}\PY{p}{(}\PY{n}{figsize}\PY{o}{=}\PY{p}{(}\PY{l+m+mi}{14}\PY{p}{,}\PY{l+m+mi}{7}\PY{p}{)}\PY{p}{)}
         \PY{n}{plt}\PY{o}{.}\PY{n}{subplot}\PY{p}{(}\PY{l+m+mi}{2}\PY{p}{,}\PY{l+m+mi}{2}\PY{p}{,}\PY{l+m+mi}{1}\PY{p}{)}
         \PY{n}{plt}\PY{o}{.}\PY{n}{title}\PY{p}{(}\PY{l+s+s1}{\PYZsq{}}\PY{l+s+s1}{Stationary Activities(Zoomed in)}\PY{l+s+s1}{\PYZsq{}}\PY{p}{)}
         \PY{n}{sns}\PY{o}{.}\PY{n}{distplot}\PY{p}{(}\PY{n}{df4}\PY{p}{[}\PY{l+s+s1}{\PYZsq{}}\PY{l+s+s1}{tBodyAccMagmean}\PY{l+s+s1}{\PYZsq{}}\PY{p}{]}\PY{p}{,}\PY{n}{color} \PY{o}{=} \PY{l+s+s1}{\PYZsq{}}\PY{l+s+s1}{r}\PY{l+s+s1}{\PYZsq{}}\PY{p}{,}\PY{n}{hist} \PY{o}{=} \PY{k+kc}{False}\PY{p}{,} \PY{n}{label} \PY{o}{=} \PY{l+s+s1}{\PYZsq{}}\PY{l+s+s1}{Sitting}\PY{l+s+s1}{\PYZsq{}}\PY{p}{)}
         \PY{n}{sns}\PY{o}{.}\PY{n}{distplot}\PY{p}{(}\PY{n}{df5}\PY{p}{[}\PY{l+s+s1}{\PYZsq{}}\PY{l+s+s1}{tBodyAccMagmean}\PY{l+s+s1}{\PYZsq{}}\PY{p}{]}\PY{p}{,}\PY{n}{color} \PY{o}{=} \PY{l+s+s1}{\PYZsq{}}\PY{l+s+s1}{m}\PY{l+s+s1}{\PYZsq{}}\PY{p}{,}\PY{n}{hist} \PY{o}{=} \PY{k+kc}{False}\PY{p}{,}\PY{n}{label} \PY{o}{=} \PY{l+s+s1}{\PYZsq{}}\PY{l+s+s1}{Standing}\PY{l+s+s1}{\PYZsq{}}\PY{p}{)}
         \PY{n}{sns}\PY{o}{.}\PY{n}{distplot}\PY{p}{(}\PY{n}{df6}\PY{p}{[}\PY{l+s+s1}{\PYZsq{}}\PY{l+s+s1}{tBodyAccMagmean}\PY{l+s+s1}{\PYZsq{}}\PY{p}{]}\PY{p}{,}\PY{n}{color} \PY{o}{=} \PY{l+s+s1}{\PYZsq{}}\PY{l+s+s1}{c}\PY{l+s+s1}{\PYZsq{}}\PY{p}{,}\PY{n}{hist} \PY{o}{=} \PY{k+kc}{False}\PY{p}{,} \PY{n}{label} \PY{o}{=} \PY{l+s+s1}{\PYZsq{}}\PY{l+s+s1}{Laying}\PY{l+s+s1}{\PYZsq{}}\PY{p}{)}
         \PY{n}{plt}\PY{o}{.}\PY{n}{axis}\PY{p}{(}\PY{p}{[}\PY{o}{\PYZhy{}}\PY{l+m+mf}{1.01}\PY{p}{,} \PY{o}{\PYZhy{}}\PY{l+m+mf}{0.5}\PY{p}{,} \PY{l+m+mi}{0}\PY{p}{,} \PY{l+m+mi}{35}\PY{p}{]}\PY{p}{)}
         \PY{n}{plt}\PY{o}{.}\PY{n}{legend}\PY{p}{(}\PY{n}{loc}\PY{o}{=}\PY{l+s+s1}{\PYZsq{}}\PY{l+s+s1}{center}\PY{l+s+s1}{\PYZsq{}}\PY{p}{)}
         
         \PY{n}{plt}\PY{o}{.}\PY{n}{subplot}\PY{p}{(}\PY{l+m+mi}{2}\PY{p}{,}\PY{l+m+mi}{2}\PY{p}{,}\PY{l+m+mi}{2}\PY{p}{)}
         \PY{n}{plt}\PY{o}{.}\PY{n}{title}\PY{p}{(}\PY{l+s+s1}{\PYZsq{}}\PY{l+s+s1}{Moving Activities}\PY{l+s+s1}{\PYZsq{}}\PY{p}{)}
         \PY{n}{sns}\PY{o}{.}\PY{n}{distplot}\PY{p}{(}\PY{n}{df1}\PY{p}{[}\PY{l+s+s1}{\PYZsq{}}\PY{l+s+s1}{tBodyAccMagmean}\PY{l+s+s1}{\PYZsq{}}\PY{p}{]}\PY{p}{,}\PY{n}{color} \PY{o}{=} \PY{l+s+s1}{\PYZsq{}}\PY{l+s+s1}{red}\PY{l+s+s1}{\PYZsq{}}\PY{p}{,}\PY{n}{hist} \PY{o}{=} \PY{k+kc}{False}\PY{p}{,} \PY{n}{label} \PY{o}{=} \PY{l+s+s1}{\PYZsq{}}\PY{l+s+s1}{Walking}\PY{l+s+s1}{\PYZsq{}}\PY{p}{)}
         \PY{n}{sns}\PY{o}{.}\PY{n}{distplot}\PY{p}{(}\PY{n}{df2}\PY{p}{[}\PY{l+s+s1}{\PYZsq{}}\PY{l+s+s1}{tBodyAccMagmean}\PY{l+s+s1}{\PYZsq{}}\PY{p}{]}\PY{p}{,}\PY{n}{color} \PY{o}{=} \PY{l+s+s1}{\PYZsq{}}\PY{l+s+s1}{blue}\PY{l+s+s1}{\PYZsq{}}\PY{p}{,}\PY{n}{hist} \PY{o}{=} \PY{k+kc}{False}\PY{p}{,}\PY{n}{label} \PY{o}{=} \PY{l+s+s1}{\PYZsq{}}\PY{l+s+s1}{Walking Up}\PY{l+s+s1}{\PYZsq{}}\PY{p}{)}
         \PY{n}{sns}\PY{o}{.}\PY{n}{distplot}\PY{p}{(}\PY{n}{df3}\PY{p}{[}\PY{l+s+s1}{\PYZsq{}}\PY{l+s+s1}{tBodyAccMagmean}\PY{l+s+s1}{\PYZsq{}}\PY{p}{]}\PY{p}{,}\PY{n}{color} \PY{o}{=} \PY{l+s+s1}{\PYZsq{}}\PY{l+s+s1}{green}\PY{l+s+s1}{\PYZsq{}}\PY{p}{,}\PY{n}{hist} \PY{o}{=} \PY{k+kc}{False}\PY{p}{,} \PY{n}{label} \PY{o}{=} \PY{l+s+s1}{\PYZsq{}}\PY{l+s+s1}{Walking down}\PY{l+s+s1}{\PYZsq{}}\PY{p}{)}
         \PY{n}{plt}\PY{o}{.}\PY{n}{legend}\PY{p}{(}\PY{n}{loc}\PY{o}{=}\PY{l+s+s1}{\PYZsq{}}\PY{l+s+s1}{center right}\PY{l+s+s1}{\PYZsq{}}\PY{p}{)}
         
         
         \PY{n}{plt}\PY{o}{.}\PY{n}{tight\PYZus{}layout}\PY{p}{(}\PY{p}{)}
         \PY{n}{plt}\PY{o}{.}\PY{n}{show}\PY{p}{(}\PY{p}{)}
\end{Verbatim}


    \begin{center}
    \adjustimage{max size={0.9\linewidth}{0.9\paperheight}}{output_41_0.png}
    \end{center}
    { \hspace*{\fill} \\}
    
    \hypertarget{magnitude-of-an-acceleration-can-saperate-it-well}{%
\subsubsection{3. Magnitude of an acceleration can saperate it
well}\label{magnitude-of-an-acceleration-can-saperate-it-well}}

    \begin{Verbatim}[commandchars=\\\{\}]
{\color{incolor}In [{\color{incolor}16}]:} \PY{n}{plt}\PY{o}{.}\PY{n}{figure}\PY{p}{(}\PY{n}{figsize}\PY{o}{=}\PY{p}{(}\PY{l+m+mi}{7}\PY{p}{,}\PY{l+m+mi}{7}\PY{p}{)}\PY{p}{)}
         \PY{n}{sns}\PY{o}{.}\PY{n}{boxplot}\PY{p}{(}\PY{n}{x}\PY{o}{=}\PY{l+s+s1}{\PYZsq{}}\PY{l+s+s1}{ActivityName}\PY{l+s+s1}{\PYZsq{}}\PY{p}{,} \PY{n}{y}\PY{o}{=}\PY{l+s+s1}{\PYZsq{}}\PY{l+s+s1}{tBodyAccMagmean}\PY{l+s+s1}{\PYZsq{}}\PY{p}{,}\PY{n}{data}\PY{o}{=}\PY{n}{train}\PY{p}{,} \PY{n}{showfliers}\PY{o}{=}\PY{k+kc}{False}\PY{p}{,} \PY{n}{saturation}\PY{o}{=}\PY{l+m+mi}{1}\PY{p}{)}
         \PY{n}{plt}\PY{o}{.}\PY{n}{ylabel}\PY{p}{(}\PY{l+s+s1}{\PYZsq{}}\PY{l+s+s1}{Acceleration Magnitude mean}\PY{l+s+s1}{\PYZsq{}}\PY{p}{)}
         \PY{n}{plt}\PY{o}{.}\PY{n}{axhline}\PY{p}{(}\PY{n}{y}\PY{o}{=}\PY{o}{\PYZhy{}}\PY{l+m+mf}{0.7}\PY{p}{,} \PY{n}{xmin}\PY{o}{=}\PY{l+m+mf}{0.1}\PY{p}{,} \PY{n}{xmax}\PY{o}{=}\PY{l+m+mf}{0.9}\PY{p}{,}\PY{n}{dashes}\PY{o}{=}\PY{p}{(}\PY{l+m+mi}{5}\PY{p}{,}\PY{l+m+mi}{5}\PY{p}{)}\PY{p}{,} \PY{n}{c}\PY{o}{=}\PY{l+s+s1}{\PYZsq{}}\PY{l+s+s1}{g}\PY{l+s+s1}{\PYZsq{}}\PY{p}{)}
         \PY{n}{plt}\PY{o}{.}\PY{n}{axhline}\PY{p}{(}\PY{n}{y}\PY{o}{=}\PY{o}{\PYZhy{}}\PY{l+m+mf}{0.05}\PY{p}{,} \PY{n}{xmin}\PY{o}{=}\PY{l+m+mf}{0.4}\PY{p}{,} \PY{n}{dashes}\PY{o}{=}\PY{p}{(}\PY{l+m+mi}{5}\PY{p}{,}\PY{l+m+mi}{5}\PY{p}{)}\PY{p}{,} \PY{n}{c}\PY{o}{=}\PY{l+s+s1}{\PYZsq{}}\PY{l+s+s1}{m}\PY{l+s+s1}{\PYZsq{}}\PY{p}{)}
         \PY{n}{plt}\PY{o}{.}\PY{n}{xticks}\PY{p}{(}\PY{n}{rotation}\PY{o}{=}\PY{l+m+mi}{90}\PY{p}{)}
         \PY{n}{plt}\PY{o}{.}\PY{n}{show}\PY{p}{(}\PY{p}{)}
\end{Verbatim}


    \begin{center}
    \adjustimage{max size={0.9\linewidth}{0.9\paperheight}}{output_43_0.png}
    \end{center}
    { \hspace*{\fill} \\}
    
    \_\_ Observations\_\_: - If tAccMean is \textless{} -0.8 then the
Activities are either Standing or Sitting or Laying. - If tAccMean is
\textgreater{} -0.6 then the Activities are either Walking or
WalkingDownstairs or WalkingUpstairs. - If tAccMean \textgreater{} 0.0
then the Activity is WalkingDownstairs. - We can classify 75\% the
Acitivity labels with some errors.

    \hypertarget{position-of-gravityaccelerationcomponants-also-matters}{%
\subsubsection{4. Position of GravityAccelerationComponants also
matters}\label{position-of-gravityaccelerationcomponants-also-matters}}

    \begin{Verbatim}[commandchars=\\\{\}]
{\color{incolor}In [{\color{incolor}17}]:} \PY{n}{sns}\PY{o}{.}\PY{n}{boxplot}\PY{p}{(}\PY{n}{x}\PY{o}{=}\PY{l+s+s1}{\PYZsq{}}\PY{l+s+s1}{ActivityName}\PY{l+s+s1}{\PYZsq{}}\PY{p}{,} \PY{n}{y}\PY{o}{=}\PY{l+s+s1}{\PYZsq{}}\PY{l+s+s1}{angleXgravityMean}\PY{l+s+s1}{\PYZsq{}}\PY{p}{,} \PY{n}{data}\PY{o}{=}\PY{n}{train}\PY{p}{)}
         \PY{n}{plt}\PY{o}{.}\PY{n}{axhline}\PY{p}{(}\PY{n}{y}\PY{o}{=}\PY{l+m+mf}{0.08}\PY{p}{,} \PY{n}{xmin}\PY{o}{=}\PY{l+m+mf}{0.1}\PY{p}{,} \PY{n}{xmax}\PY{o}{=}\PY{l+m+mf}{0.9}\PY{p}{,}\PY{n}{c}\PY{o}{=}\PY{l+s+s1}{\PYZsq{}}\PY{l+s+s1}{m}\PY{l+s+s1}{\PYZsq{}}\PY{p}{,}\PY{n}{dashes}\PY{o}{=}\PY{p}{(}\PY{l+m+mi}{5}\PY{p}{,}\PY{l+m+mi}{3}\PY{p}{)}\PY{p}{)}
         \PY{n}{plt}\PY{o}{.}\PY{n}{title}\PY{p}{(}\PY{l+s+s1}{\PYZsq{}}\PY{l+s+s1}{Angle between X\PYZhy{}axis and Gravity\PYZus{}mean}\PY{l+s+s1}{\PYZsq{}}\PY{p}{,} \PY{n}{fontsize}\PY{o}{=}\PY{l+m+mi}{15}\PY{p}{)}
         \PY{n}{plt}\PY{o}{.}\PY{n}{xticks}\PY{p}{(}\PY{n}{rotation} \PY{o}{=} \PY{l+m+mi}{40}\PY{p}{)}
         \PY{n}{plt}\PY{o}{.}\PY{n}{show}\PY{p}{(}\PY{p}{)}
\end{Verbatim}


    \begin{center}
    \adjustimage{max size={0.9\linewidth}{0.9\paperheight}}{output_46_0.png}
    \end{center}
    { \hspace*{\fill} \\}
    
    \_\_ Observations\_\_: * If angleX,gravityMean \textgreater{} 0 then
Activity is Laying. * We can classify all datapoints belonging to Laying
activity with just a single if else statement.

    \begin{Verbatim}[commandchars=\\\{\}]
{\color{incolor}In [{\color{incolor}18}]:} \PY{n}{sns}\PY{o}{.}\PY{n}{boxplot}\PY{p}{(}\PY{n}{x}\PY{o}{=}\PY{l+s+s1}{\PYZsq{}}\PY{l+s+s1}{ActivityName}\PY{l+s+s1}{\PYZsq{}}\PY{p}{,} \PY{n}{y}\PY{o}{=}\PY{l+s+s1}{\PYZsq{}}\PY{l+s+s1}{angleYgravityMean}\PY{l+s+s1}{\PYZsq{}}\PY{p}{,} \PY{n}{data} \PY{o}{=} \PY{n}{train}\PY{p}{,} \PY{n}{showfliers}\PY{o}{=}\PY{k+kc}{False}\PY{p}{)}
         \PY{n}{plt}\PY{o}{.}\PY{n}{title}\PY{p}{(}\PY{l+s+s1}{\PYZsq{}}\PY{l+s+s1}{Angle between Y\PYZhy{}axis and Gravity\PYZus{}mean}\PY{l+s+s1}{\PYZsq{}}\PY{p}{,} \PY{n}{fontsize}\PY{o}{=}\PY{l+m+mi}{15}\PY{p}{)}
         \PY{n}{plt}\PY{o}{.}\PY{n}{xticks}\PY{p}{(}\PY{n}{rotation} \PY{o}{=} \PY{l+m+mi}{40}\PY{p}{)}
         \PY{n}{plt}\PY{o}{.}\PY{n}{axhline}\PY{p}{(}\PY{n}{y}\PY{o}{=}\PY{o}{\PYZhy{}}\PY{l+m+mf}{0.22}\PY{p}{,} \PY{n}{xmin}\PY{o}{=}\PY{l+m+mf}{0.1}\PY{p}{,} \PY{n}{xmax}\PY{o}{=}\PY{l+m+mf}{0.8}\PY{p}{,} \PY{n}{dashes}\PY{o}{=}\PY{p}{(}\PY{l+m+mi}{5}\PY{p}{,}\PY{l+m+mi}{3}\PY{p}{)}\PY{p}{,} \PY{n}{c}\PY{o}{=}\PY{l+s+s1}{\PYZsq{}}\PY{l+s+s1}{m}\PY{l+s+s1}{\PYZsq{}}\PY{p}{)}
         \PY{n}{plt}\PY{o}{.}\PY{n}{show}\PY{p}{(}\PY{p}{)}
\end{Verbatim}


    \begin{center}
    \adjustimage{max size={0.9\linewidth}{0.9\paperheight}}{output_48_0.png}
    \end{center}
    { \hspace*{\fill} \\}
    
    

    

    

    \hypertarget{apply-t-sne-on-the-data}{%
\section{Apply t-sne on the data}\label{apply-t-sne-on-the-data}}

    \begin{Verbatim}[commandchars=\\\{\}]
{\color{incolor}In [{\color{incolor}46}]:} \PY{k+kn}{import} \PY{n+nn}{numpy} \PY{k}{as} \PY{n+nn}{np}
         \PY{k+kn}{from} \PY{n+nn}{sklearn}\PY{n+nn}{.}\PY{n+nn}{manifold} \PY{k}{import} \PY{n}{TSNE}
         \PY{k+kn}{import} \PY{n+nn}{matplotlib}\PY{n+nn}{.}\PY{n+nn}{pyplot} \PY{k}{as} \PY{n+nn}{plt}
         \PY{k+kn}{import} \PY{n+nn}{seaborn} \PY{k}{as} \PY{n+nn}{sns}
\end{Verbatim}


    \begin{Verbatim}[commandchars=\\\{\}]
{\color{incolor}In [{\color{incolor}47}]:} \PY{c+c1}{\PYZsh{} performs t\PYZhy{}sne with different perplexity values and their repective plots..}
         
         \PY{k}{def} \PY{n+nf}{perform\PYZus{}tsne}\PY{p}{(}\PY{n}{X\PYZus{}data}\PY{p}{,} \PY{n}{y\PYZus{}data}\PY{p}{,} \PY{n}{perplexities}\PY{p}{,} \PY{n}{n\PYZus{}iter}\PY{o}{=}\PY{l+m+mi}{1000}\PY{p}{,} \PY{n}{img\PYZus{}name\PYZus{}prefix}\PY{o}{=}\PY{l+s+s1}{\PYZsq{}}\PY{l+s+s1}{t\PYZhy{}sne}\PY{l+s+s1}{\PYZsq{}}\PY{p}{)}\PY{p}{:}
                 
             \PY{k}{for} \PY{n}{index}\PY{p}{,}\PY{n}{perplexity} \PY{o+ow}{in} \PY{n+nb}{enumerate}\PY{p}{(}\PY{n}{perplexities}\PY{p}{)}\PY{p}{:}
                 \PY{c+c1}{\PYZsh{} perform t\PYZhy{}sne}
                 \PY{n+nb}{print}\PY{p}{(}\PY{l+s+s1}{\PYZsq{}}\PY{l+s+se}{\PYZbs{}n}\PY{l+s+s1}{performing tsne with perplexity }\PY{l+s+si}{\PYZob{}\PYZcb{}}\PY{l+s+s1}{ and with }\PY{l+s+si}{\PYZob{}\PYZcb{}}\PY{l+s+s1}{ iterations at max}\PY{l+s+s1}{\PYZsq{}}\PY{o}{.}\PY{n}{format}\PY{p}{(}\PY{n}{perplexity}\PY{p}{,} \PY{n}{n\PYZus{}iter}\PY{p}{)}\PY{p}{)}
                 \PY{n}{X\PYZus{}reduced} \PY{o}{=} \PY{n}{TSNE}\PY{p}{(}\PY{n}{verbose}\PY{o}{=}\PY{l+m+mi}{2}\PY{p}{,} \PY{n}{perplexity}\PY{o}{=}\PY{n}{perplexity}\PY{p}{)}\PY{o}{.}\PY{n}{fit\PYZus{}transform}\PY{p}{(}\PY{n}{X\PYZus{}data}\PY{p}{)}
                 \PY{n+nb}{print}\PY{p}{(}\PY{l+s+s1}{\PYZsq{}}\PY{l+s+s1}{Done..}\PY{l+s+s1}{\PYZsq{}}\PY{p}{)}
                 
                 \PY{c+c1}{\PYZsh{} prepare the data for seaborn         }
                 \PY{n+nb}{print}\PY{p}{(}\PY{l+s+s1}{\PYZsq{}}\PY{l+s+s1}{Creating plot for this t\PYZhy{}sne visualization..}\PY{l+s+s1}{\PYZsq{}}\PY{p}{)}
                 \PY{n}{df} \PY{o}{=} \PY{n}{pd}\PY{o}{.}\PY{n}{DataFrame}\PY{p}{(}\PY{p}{\PYZob{}}\PY{l+s+s1}{\PYZsq{}}\PY{l+s+s1}{x}\PY{l+s+s1}{\PYZsq{}}\PY{p}{:}\PY{n}{X\PYZus{}reduced}\PY{p}{[}\PY{p}{:}\PY{p}{,}\PY{l+m+mi}{0}\PY{p}{]}\PY{p}{,} \PY{l+s+s1}{\PYZsq{}}\PY{l+s+s1}{y}\PY{l+s+s1}{\PYZsq{}}\PY{p}{:}\PY{n}{X\PYZus{}reduced}\PY{p}{[}\PY{p}{:}\PY{p}{,}\PY{l+m+mi}{1}\PY{p}{]} \PY{p}{,}\PY{l+s+s1}{\PYZsq{}}\PY{l+s+s1}{label}\PY{l+s+s1}{\PYZsq{}}\PY{p}{:}\PY{n}{y\PYZus{}data}\PY{p}{\PYZcb{}}\PY{p}{)}
                 
                 \PY{c+c1}{\PYZsh{} draw the plot in appropriate place in the grid}
                 \PY{n}{sns}\PY{o}{.}\PY{n}{lmplot}\PY{p}{(}\PY{n}{data}\PY{o}{=}\PY{n}{df}\PY{p}{,} \PY{n}{x}\PY{o}{=}\PY{l+s+s1}{\PYZsq{}}\PY{l+s+s1}{x}\PY{l+s+s1}{\PYZsq{}}\PY{p}{,} \PY{n}{y}\PY{o}{=}\PY{l+s+s1}{\PYZsq{}}\PY{l+s+s1}{y}\PY{l+s+s1}{\PYZsq{}}\PY{p}{,} \PY{n}{hue}\PY{o}{=}\PY{l+s+s1}{\PYZsq{}}\PY{l+s+s1}{label}\PY{l+s+s1}{\PYZsq{}}\PY{p}{,} \PY{n}{fit\PYZus{}reg}\PY{o}{=}\PY{k+kc}{False}\PY{p}{,} \PY{n}{size}\PY{o}{=}\PY{l+m+mi}{8}\PY{p}{,}\PYZbs{}
                            \PY{n}{palette}\PY{o}{=}\PY{l+s+s2}{\PYZdq{}}\PY{l+s+s2}{Set1}\PY{l+s+s2}{\PYZdq{}}\PY{p}{,}\PY{n}{markers}\PY{o}{=}\PY{p}{[}\PY{l+s+s1}{\PYZsq{}}\PY{l+s+s1}{\PYZca{}}\PY{l+s+s1}{\PYZsq{}}\PY{p}{,}\PY{l+s+s1}{\PYZsq{}}\PY{l+s+s1}{v}\PY{l+s+s1}{\PYZsq{}}\PY{p}{,}\PY{l+s+s1}{\PYZsq{}}\PY{l+s+s1}{s}\PY{l+s+s1}{\PYZsq{}}\PY{p}{,}\PY{l+s+s1}{\PYZsq{}}\PY{l+s+s1}{o}\PY{l+s+s1}{\PYZsq{}}\PY{p}{,} \PY{l+s+s1}{\PYZsq{}}\PY{l+s+s1}{1}\PY{l+s+s1}{\PYZsq{}}\PY{p}{,}\PY{l+s+s1}{\PYZsq{}}\PY{l+s+s1}{2}\PY{l+s+s1}{\PYZsq{}}\PY{p}{]}\PY{p}{)}
                 \PY{n}{plt}\PY{o}{.}\PY{n}{title}\PY{p}{(}\PY{l+s+s2}{\PYZdq{}}\PY{l+s+s2}{perplexity : }\PY{l+s+si}{\PYZob{}\PYZcb{}}\PY{l+s+s2}{ and max\PYZus{}iter : }\PY{l+s+si}{\PYZob{}\PYZcb{}}\PY{l+s+s2}{\PYZdq{}}\PY{o}{.}\PY{n}{format}\PY{p}{(}\PY{n}{perplexity}\PY{p}{,} \PY{n}{n\PYZus{}iter}\PY{p}{)}\PY{p}{)}
                 \PY{n}{img\PYZus{}name} \PY{o}{=} \PY{n}{img\PYZus{}name\PYZus{}prefix} \PY{o}{+} \PY{l+s+s1}{\PYZsq{}}\PY{l+s+s1}{\PYZus{}perp\PYZus{}}\PY{l+s+si}{\PYZob{}\PYZcb{}}\PY{l+s+s1}{\PYZus{}iter\PYZus{}}\PY{l+s+si}{\PYZob{}\PYZcb{}}\PY{l+s+s1}{.png}\PY{l+s+s1}{\PYZsq{}}\PY{o}{.}\PY{n}{format}\PY{p}{(}\PY{n}{perplexity}\PY{p}{,} \PY{n}{n\PYZus{}iter}\PY{p}{)}
                 \PY{n+nb}{print}\PY{p}{(}\PY{l+s+s1}{\PYZsq{}}\PY{l+s+s1}{saving this plot as image in present working directory...}\PY{l+s+s1}{\PYZsq{}}\PY{p}{)}
                 \PY{n}{plt}\PY{o}{.}\PY{n}{savefig}\PY{p}{(}\PY{n}{img\PYZus{}name}\PY{p}{)}
                 \PY{n}{plt}\PY{o}{.}\PY{n}{show}\PY{p}{(}\PY{p}{)}
                 \PY{n+nb}{print}\PY{p}{(}\PY{l+s+s1}{\PYZsq{}}\PY{l+s+s1}{Done}\PY{l+s+s1}{\PYZsq{}}\PY{p}{)}
\end{Verbatim}


    \begin{Verbatim}[commandchars=\\\{\}]
{\color{incolor}In [{\color{incolor}48}]:} \PY{n}{X\PYZus{}pre\PYZus{}tsne} \PY{o}{=} \PY{n}{train}\PY{o}{.}\PY{n}{drop}\PY{p}{(}\PY{p}{[}\PY{l+s+s1}{\PYZsq{}}\PY{l+s+s1}{subject}\PY{l+s+s1}{\PYZsq{}}\PY{p}{,} \PY{l+s+s1}{\PYZsq{}}\PY{l+s+s1}{Activity}\PY{l+s+s1}{\PYZsq{}}\PY{p}{,}\PY{l+s+s1}{\PYZsq{}}\PY{l+s+s1}{ActivityName}\PY{l+s+s1}{\PYZsq{}}\PY{p}{]}\PY{p}{,} \PY{n}{axis}\PY{o}{=}\PY{l+m+mi}{1}\PY{p}{)}
         \PY{n}{y\PYZus{}pre\PYZus{}tsne} \PY{o}{=} \PY{n}{train}\PY{p}{[}\PY{l+s+s1}{\PYZsq{}}\PY{l+s+s1}{ActivityName}\PY{l+s+s1}{\PYZsq{}}\PY{p}{]}
         \PY{n}{perform\PYZus{}tsne}\PY{p}{(}\PY{n}{X\PYZus{}data} \PY{o}{=} \PY{n}{X\PYZus{}pre\PYZus{}tsne}\PY{p}{,}\PY{n}{y\PYZus{}data}\PY{o}{=}\PY{n}{y\PYZus{}pre\PYZus{}tsne}\PY{p}{,} \PY{n}{perplexities} \PY{o}{=}\PY{p}{[}\PY{l+m+mi}{2}\PY{p}{,}\PY{l+m+mi}{5}\PY{p}{,}\PY{l+m+mi}{10}\PY{p}{,}\PY{l+m+mi}{20}\PY{p}{,}\PY{l+m+mi}{50}\PY{p}{]}\PY{p}{)}
\end{Verbatim}


    \begin{Verbatim}[commandchars=\\\{\}]

performing tsne with perplexity 2 and with 1000 iterations at max
[t-SNE] Computing 7 nearest neighbors{\ldots}
[t-SNE] Indexed 7352 samples in 0.426s{\ldots}
[t-SNE] Computed neighbors for 7352 samples in 72.001s{\ldots}
[t-SNE] Computed conditional probabilities for sample 1000 / 7352
[t-SNE] Computed conditional probabilities for sample 2000 / 7352
[t-SNE] Computed conditional probabilities for sample 3000 / 7352
[t-SNE] Computed conditional probabilities for sample 4000 / 7352
[t-SNE] Computed conditional probabilities for sample 5000 / 7352
[t-SNE] Computed conditional probabilities for sample 6000 / 7352
[t-SNE] Computed conditional probabilities for sample 7000 / 7352
[t-SNE] Computed conditional probabilities for sample 7352 / 7352
[t-SNE] Mean sigma: 0.635855
[t-SNE] Computed conditional probabilities in 0.071s
[t-SNE] Iteration 50: error = 124.8017578, gradient norm = 0.0253939 (50 iterations in 16.625s)
[t-SNE] Iteration 100: error = 107.2019501, gradient norm = 0.0284782 (50 iterations in 9.735s)
[t-SNE] Iteration 150: error = 100.9872894, gradient norm = 0.0185151 (50 iterations in 5.346s)
[t-SNE] Iteration 200: error = 97.6054382, gradient norm = 0.0142084 (50 iterations in 7.013s)
[t-SNE] Iteration 250: error = 95.3084183, gradient norm = 0.0132592 (50 iterations in 5.703s)
[t-SNE] KL divergence after 250 iterations with early exaggeration: 95.308418
[t-SNE] Iteration 300: error = 4.1209540, gradient norm = 0.0015668 (50 iterations in 7.156s)
[t-SNE] Iteration 350: error = 3.2113254, gradient norm = 0.0009953 (50 iterations in 8.022s)
[t-SNE] Iteration 400: error = 2.7819963, gradient norm = 0.0007203 (50 iterations in 9.419s)
[t-SNE] Iteration 450: error = 2.5178111, gradient norm = 0.0005655 (50 iterations in 9.370s)
[t-SNE] Iteration 500: error = 2.3341548, gradient norm = 0.0004804 (50 iterations in 7.681s)
[t-SNE] Iteration 550: error = 2.1961622, gradient norm = 0.0004183 (50 iterations in 7.097s)
[t-SNE] Iteration 600: error = 2.0867445, gradient norm = 0.0003664 (50 iterations in 9.274s)
[t-SNE] Iteration 650: error = 1.9967778, gradient norm = 0.0003279 (50 iterations in 7.697s)
[t-SNE] Iteration 700: error = 1.9210005, gradient norm = 0.0002984 (50 iterations in 8.174s)
[t-SNE] Iteration 750: error = 1.8558111, gradient norm = 0.0002776 (50 iterations in 9.747s)
[t-SNE] Iteration 800: error = 1.7989457, gradient norm = 0.0002569 (50 iterations in 8.687s)
[t-SNE] Iteration 850: error = 1.7490212, gradient norm = 0.0002394 (50 iterations in 8.407s)
[t-SNE] Iteration 900: error = 1.7043383, gradient norm = 0.0002224 (50 iterations in 8.351s)
[t-SNE] Iteration 950: error = 1.6641431, gradient norm = 0.0002098 (50 iterations in 7.841s)
[t-SNE] Iteration 1000: error = 1.6279151, gradient norm = 0.0001989 (50 iterations in 5.623s)
[t-SNE] Error after 1000 iterations: 1.627915
Done..
Creating plot for this t-sne visualization..
saving this plot as image in present working directory{\ldots}

    \end{Verbatim}

    \begin{center}
    \adjustimage{max size={0.9\linewidth}{0.9\paperheight}}{output_55_1.png}
    \end{center}
    { \hspace*{\fill} \\}
    
    \begin{Verbatim}[commandchars=\\\{\}]
Done

performing tsne with perplexity 5 and with 1000 iterations at max
[t-SNE] Computing 16 nearest neighbors{\ldots}
[t-SNE] Indexed 7352 samples in 0.263s{\ldots}
[t-SNE] Computed neighbors for 7352 samples in 48.983s{\ldots}
[t-SNE] Computed conditional probabilities for sample 1000 / 7352
[t-SNE] Computed conditional probabilities for sample 2000 / 7352
[t-SNE] Computed conditional probabilities for sample 3000 / 7352
[t-SNE] Computed conditional probabilities for sample 4000 / 7352
[t-SNE] Computed conditional probabilities for sample 5000 / 7352
[t-SNE] Computed conditional probabilities for sample 6000 / 7352
[t-SNE] Computed conditional probabilities for sample 7000 / 7352
[t-SNE] Computed conditional probabilities for sample 7352 / 7352
[t-SNE] Mean sigma: 0.961265
[t-SNE] Computed conditional probabilities in 0.122s
[t-SNE] Iteration 50: error = 114.1862640, gradient norm = 0.0184120 (50 iterations in 55.655s)
[t-SNE] Iteration 100: error = 97.6535568, gradient norm = 0.0174309 (50 iterations in 12.580s)
[t-SNE] Iteration 150: error = 93.1900101, gradient norm = 0.0101048 (50 iterations in 9.180s)
[t-SNE] Iteration 200: error = 91.2315445, gradient norm = 0.0074560 (50 iterations in 10.340s)
[t-SNE] Iteration 250: error = 90.0714417, gradient norm = 0.0057667 (50 iterations in 9.458s)
[t-SNE] KL divergence after 250 iterations with early exaggeration: 90.071442
[t-SNE] Iteration 300: error = 3.5796804, gradient norm = 0.0014691 (50 iterations in 8.718s)
[t-SNE] Iteration 350: error = 2.8173938, gradient norm = 0.0007508 (50 iterations in 10.180s)
[t-SNE] Iteration 400: error = 2.4344938, gradient norm = 0.0005251 (50 iterations in 10.506s)
[t-SNE] Iteration 450: error = 2.2156141, gradient norm = 0.0004069 (50 iterations in 10.072s)
[t-SNE] Iteration 500: error = 2.0703306, gradient norm = 0.0003340 (50 iterations in 10.511s)
[t-SNE] Iteration 550: error = 1.9646366, gradient norm = 0.0002816 (50 iterations in 9.792s)
[t-SNE] Iteration 600: error = 1.8835558, gradient norm = 0.0002471 (50 iterations in 9.098s)
[t-SNE] Iteration 650: error = 1.8184001, gradient norm = 0.0002184 (50 iterations in 8.656s)
[t-SNE] Iteration 700: error = 1.7647167, gradient norm = 0.0001961 (50 iterations in 9.063s)
[t-SNE] Iteration 750: error = 1.7193680, gradient norm = 0.0001796 (50 iterations in 9.754s)
[t-SNE] Iteration 800: error = 1.6803776, gradient norm = 0.0001655 (50 iterations in 9.540s)
[t-SNE] Iteration 850: error = 1.6465144, gradient norm = 0.0001538 (50 iterations in 9.953s)
[t-SNE] Iteration 900: error = 1.6166563, gradient norm = 0.0001421 (50 iterations in 10.270s)
[t-SNE] Iteration 950: error = 1.5901035, gradient norm = 0.0001335 (50 iterations in 6.609s)
[t-SNE] Iteration 1000: error = 1.5664237, gradient norm = 0.0001257 (50 iterations in 8.553s)
[t-SNE] Error after 1000 iterations: 1.566424
Done..
Creating plot for this t-sne visualization..
saving this plot as image in present working directory{\ldots}

    \end{Verbatim}

    \begin{center}
    \adjustimage{max size={0.9\linewidth}{0.9\paperheight}}{output_55_3.png}
    \end{center}
    { \hspace*{\fill} \\}
    
    \begin{Verbatim}[commandchars=\\\{\}]
Done

performing tsne with perplexity 10 and with 1000 iterations at max
[t-SNE] Computing 31 nearest neighbors{\ldots}
[t-SNE] Indexed 7352 samples in 0.410s{\ldots}
[t-SNE] Computed neighbors for 7352 samples in 64.801s{\ldots}
[t-SNE] Computed conditional probabilities for sample 1000 / 7352
[t-SNE] Computed conditional probabilities for sample 2000 / 7352
[t-SNE] Computed conditional probabilities for sample 3000 / 7352
[t-SNE] Computed conditional probabilities for sample 4000 / 7352
[t-SNE] Computed conditional probabilities for sample 5000 / 7352
[t-SNE] Computed conditional probabilities for sample 6000 / 7352
[t-SNE] Computed conditional probabilities for sample 7000 / 7352
[t-SNE] Computed conditional probabilities for sample 7352 / 7352
[t-SNE] Mean sigma: 1.133828
[t-SNE] Computed conditional probabilities in 0.214s
[t-SNE] Iteration 50: error = 106.0169220, gradient norm = 0.0194293 (50 iterations in 24.550s)
[t-SNE] Iteration 100: error = 90.3036194, gradient norm = 0.0097653 (50 iterations in 11.936s)
[t-SNE] Iteration 150: error = 87.3132935, gradient norm = 0.0053059 (50 iterations in 11.246s)
[t-SNE] Iteration 200: error = 86.1169128, gradient norm = 0.0035844 (50 iterations in 11.864s)
[t-SNE] Iteration 250: error = 85.4133606, gradient norm = 0.0029100 (50 iterations in 11.944s)
[t-SNE] KL divergence after 250 iterations with early exaggeration: 85.413361
[t-SNE] Iteration 300: error = 3.1394315, gradient norm = 0.0013976 (50 iterations in 11.742s)
[t-SNE] Iteration 350: error = 2.4929206, gradient norm = 0.0006466 (50 iterations in 11.627s)
[t-SNE] Iteration 400: error = 2.1733041, gradient norm = 0.0004230 (50 iterations in 11.846s)
[t-SNE] Iteration 450: error = 1.9884514, gradient norm = 0.0003124 (50 iterations in 11.405s)
[t-SNE] Iteration 500: error = 1.8702440, gradient norm = 0.0002514 (50 iterations in 11.320s)
[t-SNE] Iteration 550: error = 1.7870129, gradient norm = 0.0002107 (50 iterations in 12.009s)
[t-SNE] Iteration 600: error = 1.7246909, gradient norm = 0.0001824 (50 iterations in 10.632s)
[t-SNE] Iteration 650: error = 1.6758548, gradient norm = 0.0001590 (50 iterations in 11.270s)
[t-SNE] Iteration 700: error = 1.6361949, gradient norm = 0.0001451 (50 iterations in 12.072s)
[t-SNE] Iteration 750: error = 1.6034756, gradient norm = 0.0001305 (50 iterations in 11.607s)
[t-SNE] Iteration 800: error = 1.5761518, gradient norm = 0.0001188 (50 iterations in 9.409s)
[t-SNE] Iteration 850: error = 1.5527289, gradient norm = 0.0001113 (50 iterations in 8.309s)
[t-SNE] Iteration 900: error = 1.5328671, gradient norm = 0.0001021 (50 iterations in 9.433s)
[t-SNE] Iteration 950: error = 1.5152045, gradient norm = 0.0000974 (50 iterations in 11.488s)
[t-SNE] Iteration 1000: error = 1.4999681, gradient norm = 0.0000933 (50 iterations in 10.593s)
[t-SNE] Error after 1000 iterations: 1.499968
Done..
Creating plot for this t-sne visualization..
saving this plot as image in present working directory{\ldots}

    \end{Verbatim}

    \begin{center}
    \adjustimage{max size={0.9\linewidth}{0.9\paperheight}}{output_55_5.png}
    \end{center}
    { \hspace*{\fill} \\}
    
    \begin{Verbatim}[commandchars=\\\{\}]
Done

performing tsne with perplexity 20 and with 1000 iterations at max
[t-SNE] Computing 61 nearest neighbors{\ldots}
[t-SNE] Indexed 7352 samples in 0.425s{\ldots}
[t-SNE] Computed neighbors for 7352 samples in 61.792s{\ldots}
[t-SNE] Computed conditional probabilities for sample 1000 / 7352
[t-SNE] Computed conditional probabilities for sample 2000 / 7352
[t-SNE] Computed conditional probabilities for sample 3000 / 7352
[t-SNE] Computed conditional probabilities for sample 4000 / 7352
[t-SNE] Computed conditional probabilities for sample 5000 / 7352
[t-SNE] Computed conditional probabilities for sample 6000 / 7352
[t-SNE] Computed conditional probabilities for sample 7000 / 7352
[t-SNE] Computed conditional probabilities for sample 7352 / 7352
[t-SNE] Mean sigma: 1.274335
[t-SNE] Computed conditional probabilities in 0.355s
[t-SNE] Iteration 50: error = 97.5202179, gradient norm = 0.0223863 (50 iterations in 21.168s)
[t-SNE] Iteration 100: error = 83.9500732, gradient norm = 0.0059110 (50 iterations in 17.306s)
[t-SNE] Iteration 150: error = 81.8804779, gradient norm = 0.0035797 (50 iterations in 14.258s)
[t-SNE] Iteration 200: error = 81.1615143, gradient norm = 0.0022536 (50 iterations in 14.130s)
[t-SNE] Iteration 250: error = 80.7704086, gradient norm = 0.0018108 (50 iterations in 15.340s)
[t-SNE] KL divergence after 250 iterations with early exaggeration: 80.770409
[t-SNE] Iteration 300: error = 2.6957574, gradient norm = 0.0012993 (50 iterations in 13.605s)
[t-SNE] Iteration 350: error = 2.1637220, gradient norm = 0.0005765 (50 iterations in 13.248s)
[t-SNE] Iteration 400: error = 1.9143614, gradient norm = 0.0003474 (50 iterations in 14.774s)
[t-SNE] Iteration 450: error = 1.7684202, gradient norm = 0.0002458 (50 iterations in 15.502s)
[t-SNE] Iteration 500: error = 1.6744757, gradient norm = 0.0001923 (50 iterations in 14.808s)
[t-SNE] Iteration 550: error = 1.6101606, gradient norm = 0.0001575 (50 iterations in 14.043s)
[t-SNE] Iteration 600: error = 1.5641028, gradient norm = 0.0001344 (50 iterations in 15.769s)
[t-SNE] Iteration 650: error = 1.5291905, gradient norm = 0.0001182 (50 iterations in 15.834s)
[t-SNE] Iteration 700: error = 1.5024391, gradient norm = 0.0001055 (50 iterations in 15.398s)
[t-SNE] Iteration 750: error = 1.4809053, gradient norm = 0.0000965 (50 iterations in 14.594s)
[t-SNE] Iteration 800: error = 1.4631859, gradient norm = 0.0000884 (50 iterations in 15.025s)
[t-SNE] Iteration 850: error = 1.4486470, gradient norm = 0.0000832 (50 iterations in 14.060s)
[t-SNE] Iteration 900: error = 1.4367288, gradient norm = 0.0000804 (50 iterations in 12.389s)
[t-SNE] Iteration 950: error = 1.4270191, gradient norm = 0.0000761 (50 iterations in 10.392s)
[t-SNE] Iteration 1000: error = 1.4189968, gradient norm = 0.0000787 (50 iterations in 12.355s)
[t-SNE] Error after 1000 iterations: 1.418997
Done..
Creating plot for this t-sne visualization..
saving this plot as image in present working directory{\ldots}

    \end{Verbatim}

    \begin{center}
    \adjustimage{max size={0.9\linewidth}{0.9\paperheight}}{output_55_7.png}
    \end{center}
    { \hspace*{\fill} \\}
    
    \begin{Verbatim}[commandchars=\\\{\}]
Done

performing tsne with perplexity 50 and with 1000 iterations at max
[t-SNE] Computing 151 nearest neighbors{\ldots}
[t-SNE] Indexed 7352 samples in 0.376s{\ldots}
[t-SNE] Computed neighbors for 7352 samples in 73.164s{\ldots}
[t-SNE] Computed conditional probabilities for sample 1000 / 7352
[t-SNE] Computed conditional probabilities for sample 2000 / 7352
[t-SNE] Computed conditional probabilities for sample 3000 / 7352
[t-SNE] Computed conditional probabilities for sample 4000 / 7352
[t-SNE] Computed conditional probabilities for sample 5000 / 7352
[t-SNE] Computed conditional probabilities for sample 6000 / 7352
[t-SNE] Computed conditional probabilities for sample 7000 / 7352
[t-SNE] Computed conditional probabilities for sample 7352 / 7352
[t-SNE] Mean sigma: 1.437672
[t-SNE] Computed conditional probabilities in 0.844s
[t-SNE] Iteration 50: error = 86.1525574, gradient norm = 0.0242986 (50 iterations in 36.249s)
[t-SNE] Iteration 100: error = 75.9874649, gradient norm = 0.0061005 (50 iterations in 30.453s)
[t-SNE] Iteration 150: error = 74.7072296, gradient norm = 0.0024708 (50 iterations in 28.461s)
[t-SNE] Iteration 200: error = 74.2736282, gradient norm = 0.0018644 (50 iterations in 27.735s)
[t-SNE] Iteration 250: error = 74.0722427, gradient norm = 0.0014078 (50 iterations in 26.835s)
[t-SNE] KL divergence after 250 iterations with early exaggeration: 74.072243
[t-SNE] Iteration 300: error = 2.1539080, gradient norm = 0.0011796 (50 iterations in 25.445s)
[t-SNE] Iteration 350: error = 1.7567128, gradient norm = 0.0004845 (50 iterations in 21.282s)
[t-SNE] Iteration 400: error = 1.5888531, gradient norm = 0.0002798 (50 iterations in 21.015s)
[t-SNE] Iteration 450: error = 1.4956820, gradient norm = 0.0001894 (50 iterations in 23.332s)
[t-SNE] Iteration 500: error = 1.4359720, gradient norm = 0.0001420 (50 iterations in 23.083s)
[t-SNE] Iteration 550: error = 1.3947564, gradient norm = 0.0001117 (50 iterations in 19.626s)
[t-SNE] Iteration 600: error = 1.3653858, gradient norm = 0.0000949 (50 iterations in 22.752s)
[t-SNE] Iteration 650: error = 1.3441534, gradient norm = 0.0000814 (50 iterations in 23.972s)
[t-SNE] Iteration 700: error = 1.3284039, gradient norm = 0.0000742 (50 iterations in 20.636s)
[t-SNE] Iteration 750: error = 1.3171139, gradient norm = 0.0000700 (50 iterations in 20.407s)
[t-SNE] Iteration 800: error = 1.3085558, gradient norm = 0.0000657 (50 iterations in 24.951s)
[t-SNE] Iteration 850: error = 1.3017821, gradient norm = 0.0000603 (50 iterations in 24.719s)
[t-SNE] Iteration 900: error = 1.2962619, gradient norm = 0.0000586 (50 iterations in 24.500s)
[t-SNE] Iteration 950: error = 1.2914882, gradient norm = 0.0000573 (50 iterations in 24.132s)
[t-SNE] Iteration 1000: error = 1.2874244, gradient norm = 0.0000546 (50 iterations in 22.840s)
[t-SNE] Error after 1000 iterations: 1.287424
Done..
Creating plot for this t-sne visualization..
saving this plot as image in present working directory{\ldots}

    \end{Verbatim}

    \begin{center}
    \adjustimage{max size={0.9\linewidth}{0.9\paperheight}}{output_55_9.png}
    \end{center}
    { \hspace*{\fill} \\}
    
    \begin{Verbatim}[commandchars=\\\{\}]
Done

    \end{Verbatim}

    

    

    

    

    

    


    % Add a bibliography block to the postdoc
    
    
    
    \end{document}
